\documentclass[11pt,a4paper]{article}
\usepackage{arxiv}

\title{Securing Neural Interfaces: Architecture, Threat Taxonomy, and Neural Impact Scoring for Brain-Computer Interfaces}

\author{
  Kevin L. Qi\\
  Qinnovate\\
  \texttt{kevin@qinnovate.com}
}

\date{February 2026}

\begin{document}

\maketitle

\begin{abstract}
Brain-computer interfaces (BCIs) are transitioning from experimental neuroscience tools to
commercially deployed medical devices, with companies including Neuralink, Synchron,
Blackrock Neurotech, and Paradromics advancing toward regulatory approval and new
entrants such as Merge Labs raising \$252M in seed funding~\cite{mergelabs2026}. Yet no security
framework exists that accounts for the unique risks of devices that read and write neural
signals. The Common Vulnerability Scoring System (CVSS v4.0), the industry standard for
vulnerability assessment, cannot express biological tissue damage, cognitive integrity
violations, consent boundaries, damage reversibility, or neuroplastic consequences---dimensions
critical to neural device security.

We present an integrated security framework comprising four contributions:
(1)~an \textbf{11-band hourglass architecture} mapping attack surfaces from neocortex to
wireless radio across neural, interface, and synthetic zones;
(2)~\textbf{TARA}, a threat taxonomy of 102 techniques across 15 tactics and 8 domains, each
classified by status, severity, and dual-use therapeutic potential;
(3)~\textbf{NISS}, the Neural Impact Scoring System---a CVSS v4.0 extension adding five
neural-specific metrics (Biological Impact, Cognitive Integrity, Consent Violation,
Reversibility, Neuroplasticity) designed to conform with FIRST.org's official extension mechanism; and
(4)~the \textbf{Neural Impact Chain}, a methodology mapping security vulnerabilities to
DSM-5-TR psychiatric diagnoses through a six-stage pipeline.

Analysis of all 102 techniques reveals that 96.1\% require NISS extension metrics that CVSS
cannot express. The Neural Impact Chain maps all techniques to 15 unique DSM-5-TR diagnostic
codes across 5 psychiatric clusters, with 51 techniques posing direct diagnostic risk. The
framework identifies 77 techniques (75.5\%) with confirmed, probable, or possible therapeutic analogs,
establishing a dual-use atlas where every attack mechanism that can harm neural tissue has a
corresponding clinical application. The complete framework, threat registry, and scoring
system are released as open source under the Apache 2.0 license.
\end{abstract}

\noindent\textbf{Keywords:} brain-computer interface, neurosecurity, CVSS, threat taxonomy,
neural scoring, neuroethics, DSM-5-TR, dual-use

\tableofcontents
\newpage

% ═══════════════════════════════════════════════════════════════
% Section 1: Introduction
% ═══════════════════════════════════════════════════════════════

\section{Introduction}
\label{sec:introduction}

Brain-computer interfaces (BCIs) are no longer confined to research laboratories.
Neuralink has implanted its N1 device in human patients~\cite{musk2019neuralink},
Synchron's Stentrode has demonstrated motor neuroprosthesis via neurointerventional
surgery~\cite{oxley2021stentrode}, and Blackrock Neurotech's Utah array has enabled
high-performance speech neuroprostheses~\cite{willett2023speech}. Paradromics is
advancing toward high-bandwidth cortical interfaces for paralyzed patients.
Investment in neurotechnology grew 700\% between 2014 and 2021, with the global
market projected to reach \$25 billion by 2030~\cite{unesco2025recommendation}.

Despite this rapid commercialization, no security standard exists specifically for
neural devices. Section 3305 of the Food and Drug Omnibus Reform Act of 2022
(FDORA, Pub.~L. 117-328) added Section 524B to the FD\&C Act~\cite{fda524b},
mandating cybersecurity documentation---including threat modeling, software bills
of materials, and vulnerability disclosure---in all premarket submissions for
connected medical devices. Since October 2023, the FDA enforces a
Refuse-to-Accept policy for non-compliant submissions. However, Section 524B
does not specify which threat taxonomy or scoring system to use, and the
referenced standards (CVSS, IEC 62443, AAMI TIR57, ISO 14971) contain no
provisions for the unique risks of devices that read and write neural signals.
The EU Medical Device Regulation~\cite{eumdr2017} similarly lacks neural-specific
security requirements.

\subsection{The Scoring Gap}

The Common Vulnerability Scoring System (CVSS v4.0)~\cite{first2023cvss4} is the
industry standard for assessing vulnerability severity. CVSS excels at capturing
exploitability characteristics---attack vector, complexity, privileges required,
user interaction---and information system impact across confidentiality, integrity,
and availability. However, CVSS was designed for information technology assets.
It has no concept of:

\begin{itemize}
  \item \textbf{Biological tissue damage}---seizure induction, neural tissue
        necrosis, involuntary motor activation
  \item \textbf{Cognitive integrity}---thought privacy, perception manipulation,
        identity modification
  \item \textbf{Consent boundaries}---the distinction between operating within
        consented parameters and covert neural manipulation
  \item \textbf{Damage reversibility}---IT assets can be restored from backup;
        neural tissue cannot be rebooted
  \item \textbf{Neuroplastic consequences}---prolonged adversarial stimulation can
        cause lasting structural changes to the brain
\end{itemize}

When a vulnerability in a BCI device can induce seizures, decode private thoughts,
or cause irreversible brain damage, a CVSS base score alone is insufficient.
The gap is not theoretical: our analysis of 102 BCI attack techniques shows that
96.1\% require scoring dimensions that CVSS cannot express.

\subsection{Contributions}

This paper presents an integrated security framework with four contributions:

\begin{enumerate}
  \item \textbf{The Hourglass Architecture} (Section~\ref{sec:hourglass}): An
        11-band model mapping attack surfaces across three zones---neural (7~bands,
        from neocortex to spinal cord), interface (1~band, the electrode-tissue
        boundary), and synthetic (3~bands, from analog electronics to wireless
        radio). The hourglass shape reflects a natural security chokepoint at the
        neural interface.

  \item \textbf{TARA Threat Taxonomy} (Section~\ref{sec:tara}): A registry of
        102~techniques across 15~tactics and 8~domains, each classified by
        evidence status, severity, and dual-use therapeutic potential. TARA is
        an independent taxonomy inspired by the structural methodology of MITRE
        ATT\&CK\textsuperscript{\textregistered}, tailored to the BCI domain.

  \item \textbf{NISS Scoring System} (Section~\ref{sec:niss}): The Neural Impact
        Scoring System---a CVSS v4.0 extension designed to conform with
        FIRST.org's official extension mechanism (User Guide
        \S3.11)~\cite{first2023cvss4userguide}, adding five neural-specific
        metrics. Formal registration with FIRST.org is planned as future work.

  \item \textbf{The Neural Impact Chain} (Section~\ref{sec:nic}): A methodology
        mapping security vulnerabilities to psychiatric diagnoses through a
        six-stage pipeline: technique $\to$ hourglass band $\to$ neural
        structure $\to$ cognitive function $\to$ NISS score $\to$ DSM-5-TR
        diagnostic code~\cite{apa2022dsm5tr}.
\end{enumerate}

Sections~\ref{sec:governance}--\ref{sec:case-studies} address governance alignment
with international neuroethics frameworks and present comparative case studies.
Section~\ref{sec:limitations} discusses limitations and validation gaps honestly.
The complete framework, threat registry, and scoring system are released as open
source under the Apache 2.0 license.

\section{Related Work}
\label{sec:related}

Research at the intersection of cybersecurity and neurotechnology has progressed
through foundational framing, empirical demonstration, and emerging policy response.

\paragraph{Foundational Neurosecurity.}
Denning, Matsuoka, and Kohno~\cite{kohno2009neurosecurity}
coined ``neurosecurity'' by analyzing attack surfaces in implantable
neurostimulators, identifying wireless reprogramming, battery depletion, and signal
injection as threat categories. Bonaci et al.~\cite{bonaci2015appstores} extended
this with ``App Stores for the Brain,'' examining privacy threats from third-party
BCI applications.

\paragraph{Empirical Attacks.}
Martinovic et al.~\cite{martinovic2012feasibility} demonstrated at USENIX Security
2012 that consumer EEG headsets could be exploited via P300 event-related potentials
to extract PINs and personal information through subliminal visual stimuli. This
remains the most influential empirical BCI side-channel attack.
Meng et al.~\cite{meng2023adversarial} established an adversarial robustness
benchmark for EEG-based BCIs, evaluating multiple defense approaches across
neural network architectures and datasets. Halperin et al.~\cite{halperin2008pacemakers}
established that implanted medical device attacks are practical by demonstrating
wireless attacks against pacemakers.

\paragraph{Recent Frameworks.}
Schroder et al.~\cite{schroder2025cyberrisks} published the most recent analysis
(2025), developing a threat model for network-based attacks on next-generation BCIs
and identifying associated regulatory changes. Ienca and Haselager~\cite{ienca2016hacking} analyzed BCI security through
informational and physical integrity, proposing that BCIs require both cybersecurity
and neuroethical safeguards. Camara et al.~\cite{camara2015security} and Rushanan
et al.~\cite{rushanan2014sok} surveyed security in implantable medical devices
broadly, covering pacemakers and neurostimulators.

\paragraph{Neuroethics and Policy.}
Ienca and Andorno~\cite{ienca2017neurorights} proposed four neurorights (cognitive
liberty, mental privacy, mental integrity, psychological continuity). UNESCO's
2025 Recommendation~\cite{unesco2025recommendation}, adopted by 194 Member States,
is the first global normative framework for neurotechnology governance.

\paragraph{Gap.}
Prior work addresses BCI security, neuroethics, and vulnerability scoring
separately. No existing framework integrates architecture, technique-level
taxonomy, CVSS-compatible scoring, and clinical impact mapping into a single
system. The vulnerability-to-diagnosis pipeline has no precedent in either
cybersecurity or neuroethics literature.

% ═══════════════════════════════════════════════════════════════
% Section 3: The Hourglass Architecture
% ═══════════════════════════════════════════════════════════════

\section{The Hourglass Architecture}
\label{sec:hourglass}

The \qif architecture maps the full BCI attack surface using an 11-band hourglass
model organized into three zones. The model is derived from two independent
design traditions: the OSI networking reference model~\cite{zimmermann1980osi},
which stratifies communication systems into functional layers, and functional
neuroanatomy, which organizes the nervous system by anatomical structure and
physiological role. The hourglass shape borrows from the Internet protocol
architecture~\cite{deering1998hourglass}, where IP serves as a narrow waist
through which all traffic must pass.

\subsection{Design Rationale}

A BCI system spans from high-level cortical computation to low-level radio
transmission. Any security framework must account for threats at every point
along this path. We observe that the neural interface---the electrode-tissue
boundary---forms a natural chokepoint analogous to IP in the Internet hourglass:
all signals, whether neural or synthetic, must cross this boundary. Above the
interface, attack surfaces expand through the complexity of neural tissue. Below
it, they expand through electronic and wireless systems. The resulting shape is
an asymmetric hourglass with the narrowest point at the interface.

\subsection{Band Definitions}

Table~\ref{tab:bands} defines all 11 bands. The model uses a 7-1-3 asymmetric
structure: seven neural bands (N7--N1), one interface band (I0), and three
synthetic bands (S1--S3).

\begin{table}[H]
\centering
\caption{QIF Hourglass: 11 bands across three zones.}
\label{tab:bands}
\small
\begin{tabularx}{\textwidth}{@{}l l l X@{}}
\toprule
\textbf{Band} & \textbf{Name} & \textbf{Zone} & \textbf{Attack Surface} \\
\midrule
N7 & Neocortex         & Neural     & PFC, M1, V1, Broca, Wernicke---executive function, language, movement, perception \\
N6 & Limbic System     & Neural     & Hippocampus, amygdala, insula---emotion, memory, interoception \\
N5 & Basal Ganglia     & Neural     & Striatum, STN, substantia nigra---motor selection, reward, habit \\
N4 & Diencephalon      & Neural     & Thalamus, hypothalamus---sensory gating, consciousness relay \\
N3 & Cerebellum        & Neural     & Cerebellar cortex, deep nuclei---motor coordination, timing \\
N2 & Brainstem         & Neural     & Medulla, pons, midbrain---vital functions, arousal, reflexes \\
N1 & Spinal Cord       & Neural     & Cervical through sacral---reflexes, peripheral relay \\
\midrule
I0 & Neural Interface  & Interface  & Electrode-tissue boundary---the hourglass waist; all signals must cross \\
\midrule
S1 & Analog / Near-Field & Synthetic & Amplification, ADC, near-field EM (0--10\,kHz) \\
S2 & Digital / Telemetry & Synthetic & Decoding, BLE/WiFi, telemetry (10\,kHz--1\,GHz) \\
S3 & Radio / Wireless    & Synthetic & RF, directed energy, application layer (1\,GHz+) \\
\bottomrule
\end{tabularx}
\end{table}

\subsection{Zone Structure}

\paragraph{Neural Zone (N7--N1).}
The neural zone encompasses all biological structures from the neocortex to the
spinal cord. Attack surfaces in this zone involve direct interaction with neural
tissue: signal injection, neural manipulation, cognitive exploitation, and
physical safety threats. The ordering follows neuroanatomical hierarchy---higher
bands represent more complex cognitive functions, while lower bands represent
more fundamental physiological processes. Attacks at lower neural bands (N1--N2)
threaten vital functions; attacks at higher bands (N6--N7) threaten cognition,
identity, and autonomy.

\paragraph{Interface Zone (I0).}
Band I0 is the singular chokepoint where biological and synthetic signals
convert. For invasive BCIs, this is the electrode-tissue boundary where
electrical signals are transduced from ionic currents in neural tissue to
electronic currents in silicon. For non-invasive devices, it is the sensor
surface (e.g., scalp electrodes for EEG). The interface band is the narrowest
point in the hourglass and the most critical for security: every neural signal
that reaches the synthetic system, and every stimulation signal that reaches
neural tissue, must pass through I0.

\paragraph{Synthetic Zone (S1--S3).}
The synthetic zone covers electronic and wireless systems from the analog
front-end through digital processing to radio transmission. S1 handles analog
signal conditioning and near-field electromagnetic effects. S2 encompasses
digital processing, Bluetooth Low Energy, WiFi, and telemetry protocols.
S3 covers radio frequency transmission, directed energy, and application-layer
communication. Threats in this zone are closest to traditional IT security
and are most amenable to conventional countermeasures.

\subsection{Security Properties}

The hourglass architecture provides three properties relevant to security analysis:

\begin{enumerate}
  \item \textbf{Completeness}: Every component of a BCI system maps to exactly
        one band. There is no attack surface that falls outside the model.

  \item \textbf{Chokepoint identification}: Band I0 identifies the natural
        monitoring point where all signals can be inspected during transit between
        neural and synthetic zones.

  \item \textbf{Threat locality}: Each technique in the TARA taxonomy
        (Section~\ref{sec:tara}) is annotated with the bands it affects, enabling
        defenders to understand the spatial extent of an attack across the
        architecture.
\end{enumerate}

Figure~\ref{fig:hourglass} illustrates the 11-band model with relative band
widths reflecting attack surface breadth at each layer.

\begin{figure}[H]
  \centering
  \includegraphics[width=0.6\textwidth]{figures/hourglass.pdf}
  \caption{The QIF Hourglass Architecture. Band widths reflect relative attack
  surface breadth. The neural interface (I0) forms the narrow waist---the
  security chokepoint through which all signals must pass.}
  \label{fig:hourglass}
\end{figure}

% ═══════════════════════════════════════════════════════════════
% Section 4: TARA Threat Taxonomy
% ═══════════════════════════════════════════════════════════════

\section{TARA: Threat Taxonomy}
\label{sec:tara}

The Therapeutic Applications \& Risk Assessment (TARA) registry catalogs BCI
attack techniques with a structure inspired by MITRE
ATT\&CK~\cite{mitre2024attack}, independently developed to cover neural,
cognitive, and physical safety domains. Each
technique is simultaneously an attack vector, an ethical risk, and---where
applicable---a therapeutic application.

\paragraph{Why not ATT\&CK directly?}
MITRE ATT\&CK is designed for enterprise IT and mobile environments where
adversary objectives center on data exfiltration, lateral movement, and
persistence. BCI threats differ in three fundamental ways that make direct
adoption inadequate: (1)~the target is biological tissue, not an information
system---attacks can cause seizures, tissue necrosis, or cognitive coercion,
none of which map to ATT\&CK's impact categories; (2)~the same technique is
often simultaneously an attack and a therapy (e.g., deep brain stimulation is
both a Parkinson's treatment and a neural injection vector), requiring a
dual-use classification that ATT\&CK has no mechanism to express; and (3)~the
attack lifecycle must span from radio-frequency transmission through silicon
processing to neural tissue, crossing the bio-digital boundary that ATT\&CK's
purely digital model does not address. TARA adopts ATT\&CK's proven structural
methodology---techniques organized into tactics and domains---while building an
independent taxonomy purpose-built for the BCI threat landscape.

\subsection{Taxonomy Structure}

The TARA registry organizes 102 techniques into 15 tactics across 8 operational
domains. Technique identifiers follow the format \texttt{QIF-TXXX}, where the
numeric suffix provides sequential ordering.

\paragraph{Domains.}
The eight domains span the full attack lifecycle:

\begin{enumerate}
  \item \textbf{Neural (N)} --- Direct interaction with neural tissue
        (scan, injection, manipulation)
  \item \textbf{BCI System (B)} --- System-level intrusion and evasion
  \item \textbf{Protocol (P)} --- Protocol disruption and communication attacks
  \item \textbf{Data (D)} --- Data harvesting and exfiltration
  \item \textbf{Cognitive (C)} --- Cognitive exploitation and imprinting
  \item \textbf{Countermeasure (M)} --- Surveillance and monitoring
  \item \textbf{Evasion (E)} --- Defense evasion and anti-detection
  \item \textbf{Sensor (S)} --- Consumer device side-channel attacks
\end{enumerate}

\paragraph{Tactics.}
The 15 tactics follow a lifecycle structure analogous to ATT\&CK, adapted for BCI operations:
Neural Scan, BCI Intrusion, Neural Injection, Cognitive Imprinting, BCI Evasion,
Data Harvesting, Neural Manipulation, Evasion/Rootkit Deployment, Monitoring/Surveillance,
Protocol Disruption, Cognitive Exploitation, Signal Replay, Signal Harvesting,
Signal Fingerprinting, and Signal Chaining.

\subsection{Evidence Classification}

Each technique carries an evidence status reflecting the maturity of its
documentation:

\begin{table}[H]
\centering
\caption{TARA evidence status classification with technique counts.}
\label{tab:status}
\small
\begin{tabular}{@{}l l r@{}}
\toprule
\textbf{Status} & \textbf{Definition} & \textbf{Count} \\
\midrule
Confirmed    & Documented in real-world use or peer-reviewed literature & 19 \\
Demonstrated & Proven in laboratory or controlled conditions            & 33 \\
Emerging     & Newly identified; limited but growing evidence           & 22 \\
Theoretical  & Plausible based on known physics and engineering          & 26 \\
Plausible    & Possible but with significant uncertainty                & 1 \\
Speculative  & Hypothetical; requires unproven capabilities             & 1 \\
\bottomrule
\end{tabular}
\end{table}

\subsection{Severity Distribution}

CVSS v4.0 base severity ratings for all 102 techniques show a distribution
heavily weighted toward high and critical:

\begin{table}[H]
\centering
\caption{TARA severity distribution (CVSS v4.0 base ratings).}
\label{tab:severity}
\small
\begin{tabular}{@{}l r r@{}}
\toprule
\textbf{Severity} & \textbf{Count} & \textbf{Percentage} \\
\midrule
Critical & 29 & 28.4\% \\
High     & 54 & 52.9\% \\
Medium   & 16 & 15.7\% \\
Low      &  3 &  2.9\% \\
\bottomrule
\end{tabular}
\end{table}

The dominance of high and critical ratings reflects the inherent severity of
attacks against devices that interface directly with the nervous system. Even
techniques with low exploitability can have severe consequences when the target
is neural tissue.

\subsection{Category Breakdown}

The 102 techniques distribute across eight operational categories:

\begin{table}[H]
\centering
\caption{TARA techniques by operational category.}
\label{tab:categories}
\small
\begin{tabular}{@{}l l r@{}}
\toprule
\textbf{ID} & \textbf{Category} & \textbf{Count} \\
\midrule
SE & Signal Eavesdropping    & 20 \\
CI & Cognitive Integrity     & 18 \\
EX & Data Exfiltration       & 17 \\
DM & Data Manipulation       & 15 \\
SI & Signal Injection        & 10 \\
PE & Privilege Escalation    &  8 \\
DS & Denial of Service       &  7 \\
PS & Physical Safety         &  7 \\
\bottomrule
\end{tabular}
\end{table}

\subsection{Dual-Use Mapping}

A distinctive feature of TARA is the systematic dual-use mapping: every attack
technique is assessed for therapeutic analogs. The same physical mechanisms that
enable attacks---electromagnetic stimulation, signal decoding, neuromodulation---are
the mechanisms underlying established therapies such as deep brain stimulation
(DBS)~\cite{lozano2019dbs}, transcranial magnetic stimulation
(TMS)~\cite{hallett2007tms}, and neurofeedback.

\begin{table}[H]
\centering
\caption{Dual-use classification of 102 TARA techniques.}
\label{tab:dualuse}
\small
\begin{tabular}{@{}l l r r@{}}
\toprule
\textbf{Classification} & \textbf{Definition} & \textbf{Count} & \textbf{\%} \\
\midrule
Confirmed    & Published clinical use exists      & 52 & 51.0\% \\
Probable     & Under active clinical investigation & 16 & 15.7\% \\
Possible     & Theoretical therapeutic mapping     &  9 &  8.8\% \\
Silicon Only & No tissue analog; purely digital    & 25 & 24.5\% \\
\bottomrule
\end{tabular}
\end{table}

Of the 102 techniques, 77 (75.5\%) have confirmed, probable, or possible therapeutic analogs.
This finding underscores a fundamental challenge for BCI security: the same
capabilities that must be defended against are often the capabilities that make
BCIs therapeutically valuable.

\subsection{Representative Techniques}

Table~\ref{tab:representative} shows five representative techniques spanning
different categories, severities, and dual-use classifications.

\begin{table}[H]
\centering
\caption{Five representative TARA techniques.}
\label{tab:representative}
\small
\begin{tabularx}{\textwidth}{@{}l l l l X@{}}
\toprule
\textbf{ID} & \textbf{Sev.} & \textbf{Status} & \textbf{Dual-Use} & \textbf{Description} \\
\midrule
T0001 & Critical & Confirmed & Confirmed & Cortical signal injection via rogue electrode stimulation; therapeutic analog: deep brain stimulation \\
T0015 & High & Demonstrated & Confirmed & P300 side-channel extraction of private information; therapeutic analog: P300-based spelling interfaces \\
T0034 & High & Emerging & Probable & Calibration data poisoning during BCI training sessions; therapeutic analog: adaptive neurofeedback \\
T0072 & Medium & Confirmed & Confirmed & Ultrasonic side-channel via bone conduction microphone; therapeutic analog: ABR audiometry \\
T0090 & High & Demonstrated & Confirmed & WiFi CSI body sensing for respiratory and gait inference; therapeutic analog: sleep apnea detection \\
\bottomrule
\end{tabularx}
\end{table}

Figure~\ref{fig:severity-dist} shows the severity distribution across all
techniques, and Figure~\ref{fig:dual-use-breakdown} illustrates the dual-use
breakdown.

\begin{figure}[H]
  \centering
  \begin{minipage}[t]{0.48\textwidth}
    \centering
    \includegraphics[width=\textwidth]{figures/severity-dist.pdf}
    \caption{TARA severity distribution across 102 techniques.}
    \label{fig:severity-dist}
  \end{minipage}
  \hfill
  \begin{minipage}[t]{0.48\textwidth}
    \centering
    \includegraphics[width=\textwidth]{figures/dual-use-breakdown.pdf}
    \caption{Dual-use classification: 75.5\% of techniques have therapeutic analogs.}
    \label{fig:dual-use-breakdown}
  \end{minipage}
\end{figure}

% ═══════════════════════════════════════════════════════════════
% Section 5: NISS — Neural Impact Scoring System
% ═══════════════════════════════════════════════════════════════

\section{NISS: Neural Impact Scoring System}
\label{sec:niss}

The Neural Impact Scoring System (NISS) is a CVSS v4.0 extension following
FIRST.org's official extension mechanism (User Guide
\S3.11)~\cite{first2023cvss4userguide}. NISS adds five metrics that capture
dimensions CVSS was never designed to express: biological tissue damage, cognitive
integrity violations, consent boundary violations, damage reversibility, and
neuroplastic consequences.

\subsection{Gap Analysis: Why CVSS Alone Is Insufficient}

We mapped all 102 TARA techniques to CVSS v4.0 base vectors and classified the
results into three gap groups based on how much information CVSS alone fails to
capture:

\begin{table}[H]
\centering
\caption{CVSS v4.0 gap analysis across 102 TARA techniques.}
\label{tab:gap}
\small
\begin{tabularx}{\textwidth}{@{}c X r l@{}}
\toprule
\textbf{Group} & \textbf{Gap Description} & \textbf{Count} & \textbf{Example} \\
\midrule
1 & CVSS captures most impact; NISS adds nuance &
    12 & Digital-only \\
2 & CVSS captures exploitability but misses half &
    28 & Mixed impact \\
3 & CVSS fundamentally cannot express primary impact &
    58 & Neural-dominant \\
\midrule
  & \textbf{Techniques needing NISS extension} & \textbf{98} & \textbf{96.1\%} \\
\bottomrule
\end{tabularx}
\end{table}

Group~3---where CVSS fundamentally cannot express the primary impact---contains
the majority of techniques (56.9\%). These are attacks where the most severe
consequence is biological tissue damage, cognitive coercion, or irreversible
neural harm---dimensions for which CVSS has no metric.

\subsection{Extension Metrics}

NISS defines five extension metrics, each with a graduated value set. The
metrics are designed to be orthogonal to CVSS base metrics: they capture impact
dimensions that exist only because the target system interfaces with biological
neural tissue.

\subsubsection{BI: Biological Impact}

Direct harm to neural tissue, organs, or physiological function. This dimension
has no equivalent in CVSS.

\begin{table}[H]
\centering
\small
\begin{tabular}{@{}l l l p{7cm}@{}}
\toprule
\textbf{Value} & \textbf{Label} & \textbf{Score} & \textbf{Description} \\
\midrule
N & None     & 0.0  & No tissue interaction or physical harm \\
L & Low      & 3.3  & Temporary discomfort, minor sensory disruption, reversible tissue stress \\
H & High     & 6.7  & Significant tissue damage, seizure induction, involuntary motor activation \\
C & Critical & 10.0 & Life-threatening or permanently disabling neural harm \\
\bottomrule
\end{tabular}
\end{table}

\subsubsection{CG: Cognitive Integrity}

Impact on thought processes, perception, memory, identity, or decision-making.
CVSS has no concept of thought privacy or cognitive autonomy.

\begin{table}[H]
\centering
\small
\begin{tabular}{@{}l l l p{7cm}@{}}
\toprule
\textbf{Value} & \textbf{Label} & \textbf{Score} & \textbf{Description} \\
\midrule
N & None     & 0.0  & No cognitive impact \\
L & Low      & 3.3  & Decoded intent partially exposed, minor perceptual distortion \\
H & High     & 6.7  & Full thought decoding, identity inference, or perception manipulation \\
C & Critical & 10.0 & Cognitive coercion, identity modification, or complete loss of cognitive autonomy \\
\bottomrule
\end{tabular}
\end{table}

\subsubsection{CV: Consent Violation}

Degree of violation of informed consent or cognitive autonomy. Ordered by
severity: covert (implicit) violations are worse than detectable (explicit) ones.

\begin{table}[H]
\centering
\small
\begin{tabular}{@{}l l l p{7cm}@{}}
\toprule
\textbf{Value} & \textbf{Label} & \textbf{Score} & \textbf{Description} \\
\midrule
N & None              & 0.0  & Operating within explicitly consented boundaries \\
P & Partial           & 3.3  & Action exceeds scope but subject retains some awareness \\
E & Explicit          & 6.7  & Direct violation of consent boundaries, but detectable \\
I & Implicit (covert) & 10.0 & Covert manipulation the patient cannot detect or refuse \\
\bottomrule
\end{tabular}
\end{table}

\subsubsection{RV: Reversibility}

Whether the damage can be undone. IT assets can be restored from backup. Neural
tissue cannot be rebooted.

\begin{table}[H]
\centering
\small
\begin{tabular}{@{}l l l p{7cm}@{}}
\toprule
\textbf{Value} & \textbf{Label} & \textbf{Score} & \textbf{Description} \\
\midrule
F & Full        & 0.0  & Effects fully reverse when attack stops \\
T & Temporary   & 3.3  & Effects reverse over hours to days \\
P & Partial     & 6.7  & Some effects permanent, some reversible \\
I & Irreversible & 10.0 & Permanent neural tissue destruction or cognitive change \\
\bottomrule
\end{tabular}
\end{table}

\subsubsection{NP: Neuroplasticity}

Whether the attack exploits or induces neuroplastic changes---the brain's ability
to rewire itself. This has no digital equivalent.

\begin{table}[H]
\centering
\small
\begin{tabular}{@{}l l l p{7cm}@{}}
\toprule
\textbf{Value} & \textbf{Label} & \textbf{Score} & \textbf{Description} \\
\midrule
N & None       & 0.0  & No neuroplastic effect \\
T & Temporary  & 5.0  & Short-term synaptic changes that decay within hours to days \\
S & Structural & 10.0 & Long-term or permanent neural pathway changes \\
\bottomrule
\end{tabular}
\end{table}

\subsection{PINS Flag}

NISS introduces the Persistent Involuntary Neural Stimulation (PINS) flag---a
binary indicator triggered when:

\begin{equation}
\text{PINS} = \begin{cases}
  \texttt{true} & \text{if } \text{BI} \geq \text{High} \;\lor\; \text{RV} = \text{Irreversible} \\
  \texttt{false} & \text{otherwise}
\end{cases}
\end{equation}

A PINS flag mandates immediate safety review regardless of overall score.
Across all 102 techniques, 31 are PINS-flagged (30.4\%).

\subsection{Scoring Formula}

The NISS score is computed as the weighted mean of the five metric scores:

\begin{equation}
\text{NISS} = \frac{w_{\text{BI}} \cdot \text{BI} + w_{\text{CG}} \cdot \text{CG} + w_{\text{CV}} \cdot \text{CV} + w_{\text{RV}} \cdot \text{RV} + w_{\text{NP}} \cdot \text{NP}}
              {w_{\text{BI}} + w_{\text{CG}} + w_{\text{CV}} + w_{\text{RV}} + w_{\text{NP}}}
\label{eq:niss}
\end{equation}

In the default profile, all weights are~1.0, yielding a simple arithmetic mean.
NISS supports four context profiles with differential weights:

\begin{itemize}
  \item \textbf{Clinical}: Emphasizes BI, RV, and NP (patient safety focus)
  \item \textbf{Research}: Emphasizes CG and CV (consent and cognition focus)
  \item \textbf{Consumer}: Balanced weights (general-purpose)
  \item \textbf{Military}: Emphasizes BI and CG (dual-use concern)
\end{itemize}

\subsection{Vector Format}

The NISS vector rides alongside the CVSS v4.0 base vector:

\begin{lstlisting}
CVSS:4.0/AV:N/AC:L/AT:N/PR:N/UI:N/
  VC:H/VI:H/VA:H/SC:N/SI:N/SA:N
NISS:1.0/BI:H/CG:C/CV:I/RV:P/NP:S
\end{lstlisting}

This dual-vector architecture means security teams can triage using familiar CVSS
scores while BCI-specific teams see the neural dimensions that determine whether a
vulnerability is a software bug or a patient safety emergency.

\subsection{NISS Severity Distribution}

Across all 102 techniques, the NISS severity distribution differs markedly from
CVSS severity:

\begin{table}[H]
\centering
\caption{NISS severity distribution (all 102 techniques).}
\label{tab:niss-severity}
\small
\begin{tabular}{@{}l r r@{}}
\toprule
\textbf{NISS Severity} & \textbf{Count} & \textbf{\%} \\
\midrule
High     & 21 & 20.6\% \\
Medium   & 29 & 28.4\% \\
Low      & 51 & 50.0\% \\
None     &  1 &  1.0\% \\
\bottomrule
\end{tabular}
\end{table}

The NISS distribution is more uniform than CVSS because NISS captures impact
dimensions that CVSS flattens into high/critical. Techniques rated ``high'' by
CVSS may distribute across low, medium, and high NISS scores depending on whether
they involve biological tissue damage or are purely digital.

Figure~\ref{fig:niss-gap} visualizes the gap between CVSS and NISS scoring.

\begin{figure}[H]
  \centering
  \includegraphics[width=0.7\textwidth]{figures/niss-gap.pdf}
  \caption{CVSS v4.0 vs.\ NISS gap analysis: 96.1\% of techniques require
  extension metrics CVSS cannot express.}
  \label{fig:niss-gap}
\end{figure}

% ═══════════════════════════════════════════════════════════════
% Section 6: Neural Impact Chain
% ═══════════════════════════════════════════════════════════════

\section{Neural Impact Chain}
\label{sec:nic}

The Neural Impact Chain (NIC) is a six-stage methodology for mapping security
vulnerabilities to clinical psychiatric diagnoses. To our knowledge, this is the
first systematic pipeline connecting cybersecurity severity scoring to DSM-5-TR
diagnostic codes~\cite{apa2022dsm5tr}. The NIC answers a question no prior
framework has addressed: \emph{if this attack succeeds, what psychiatric condition
could it cause or worsen?}

\subsection{Pipeline Architecture}

The NIC traces each technique through six stages:

\begin{enumerate}
  \item \textbf{Technique}: The TARA attack technique (e.g., cortical signal
        injection)
  \item \textbf{Hourglass Band}: Which band(s) the technique affects (e.g., N7
        Neocortex, N6 Limbic)
  \item \textbf{Neural Structure}: The anatomical structure at risk (e.g.,
        hippocampus, prefrontal cortex, amygdala)
  \item \textbf{Cognitive Function}: The function that structure supports (e.g.,
        memory consolidation, executive control, emotional regulation)
  \item \textbf{NISS Score}: The neural impact score, particularly the BI, CG,
        and NP metrics that correlate with clinical outcomes
  \item \textbf{DSM-5-TR Code}: The ICD-10-CM diagnostic code(s) for the
        psychiatric condition most closely associated with disruption of that
        function
\end{enumerate}

Figure~\ref{fig:nic} illustrates this pipeline.

\begin{figure}[H]
  \centering
  \includegraphics[width=0.85\textwidth]{figures/neural-impact-chain.pdf}
  \caption{The Neural Impact Chain: six-stage pipeline from security technique
  to psychiatric diagnosis. Each arrow represents a mapping grounded in
  neuroanatomy, functional neuroscience, or clinical psychiatry.}
  \label{fig:nic}
\end{figure}

\subsection{NISS-to-DSM Bridge}

The bridge between NISS metrics and DSM-5-TR diagnostic clusters is driven by
which NISS metric dominates the technique's profile:

\begin{itemize}
  \item \textbf{BI-driven} $\to$ Motor/Neurocognitive cluster: Biological
        impact implies tissue damage, leading to movement disorders or
        neurocognitive deficits
  \item \textbf{CG-driven} $\to$ Cognitive/Psychotic cluster: Cognitive
        integrity violations imply perceptual or thought-process disruption,
        leading to psychotic or dissociative symptoms
  \item \textbf{CV-driven} $\to$ Mood/Trauma cluster: Consent violations
        imply autonomy loss, mapping to trauma- and stressor-related disorders
  \item \textbf{NP/RV-driven} $\to$ Persistent/Personality cluster:
        Neuroplasticity exploitation or irreversible damage implies lasting
        personality or behavioral changes
  \item \textbf{No neural impact} $\to$ Non-diagnostic: Silicon-only
        techniques with no direct psychiatric mapping
\end{itemize}

\subsection{Coverage Statistics}

All 102 TARA techniques have been mapped through the NIC pipeline:

\begin{table}[H]
\centering
\caption{Neural Impact Chain mapping results across 102 techniques.}
\label{tab:nic-stats}
\small
\begin{tabular}{@{}l r@{}}
\toprule
\textbf{Metric} & \textbf{Value} \\
\midrule
Techniques mapped                 & 102 / 102 (100\%) \\
Unique DSM-5-TR codes             & 15 \\
Diagnostic clusters               & 5 \\
Direct diagnostic risk             & 51 (50.0\%) \\
Indirect diagnostic risk           & 9 (8.8\%) \\
No diagnostic risk (silicon-only)  & 42 (41.2\%) \\
\bottomrule
\end{tabular}
\end{table}

\subsection{Diagnostic Cluster Distribution}

The five diagnostic clusters and their technique counts:

\begin{table}[H]
\centering
\caption{DSM-5-TR diagnostic cluster distribution.}
\label{tab:clusters}
\small
\begin{tabular}{@{}l r l@{}}
\toprule
\textbf{Cluster} & \textbf{Count} & \textbf{Representative DSM-5-TR Codes} \\
\midrule
Non-diagnostic          & 42 & --- (silicon-only techniques) \\
Mood/Trauma             & 21 & F43.10 (PTSD), F32.9 (MDD), F44.9 (dissociative) \\
Cognitive/Psychotic     & 16 & F06.0 (psychosis due to medical condition), R41.3 (cognitive decline) \\
Motor/Neurocognitive    & 16 & G25.9 (movement disorder), G31.84 (neurocognitive) \\
Persistent/Personality  &  7 & F07.0 (personality change due to medical condition) \\
\bottomrule
\end{tabular}
\end{table}

The Mood/Trauma cluster is the largest diagnostic cluster (21 techniques),
reflecting the prevalence of consent-violation and autonomy-disruption attacks
in the BCI threat landscape. Techniques that covertly manipulate neural signals
without the subject's knowledge or consent map naturally to trauma- and
stressor-related disorders.

\subsection{Risk Classification}

Each technique is classified by diagnostic risk:

\begin{itemize}
  \item \textbf{Direct} (51 techniques, 50.0\%): The attack mechanism can
        directly trigger or worsen the mapped psychiatric condition. Example:
        forced cortical stimulation causing seizures maps to epilepsy-related
        diagnostic codes.
  \item \textbf{Indirect} (9 techniques, 8.8\%): The attack creates downstream
        conditions that may lead to the diagnosis. Example: sustained data
        exfiltration of private thoughts causing anxiety does not directly
        produce the anxiety disorder but creates the conditions for it.
  \item \textbf{None} (42 techniques, 41.2\%): Silicon-only techniques with no
        direct neural interaction and thus no psychiatric diagnostic mapping.
\end{itemize}

\subsection{Example Walkthrough}

Consider technique \textbf{QIF-T0001: Cortical Signal Injection}---direct
injection of adversarial signals via rogue electrode stimulation.

\begin{enumerate}
  \item \textbf{Technique}: QIF-T0001 (cortical signal injection)
  \item \textbf{Band}: N7 (Neocortex), N6 (Limbic System)
  \item \textbf{Structure}: Primary motor cortex (M1), prefrontal cortex (PFC),
        hippocampus
  \item \textbf{Function}: Motor control, executive function, memory
        consolidation
  \item \textbf{NISS}: BI:C / CG:H / CV:I / RV:P / NP:S $\to$ Score: 8.7
        (High), PINS flagged
  \item \textbf{DSM-5-TR}: G25.9 (movement disorder NOS), F06.0 (psychotic
        disorder due to another medical condition), F07.0 (personality change
        due to another medical condition)
  \item \textbf{Risk class}: Direct---the stimulation itself can trigger
        seizures, involuntary movement, and perception distortion
\end{enumerate}

Figure~\ref{fig:dsm5-clusters} shows the distribution of techniques across
diagnostic clusters.

\begin{figure}[H]
  \centering
  \includegraphics[width=0.65\textwidth]{figures/dsm5-clusters.pdf}
  \caption{DSM-5-TR cluster distribution across 102 TARA techniques. 60
  techniques (58.8\%) have direct or indirect diagnostic risk.}
  \label{fig:dsm5-clusters}
\end{figure}

% ═══════════════════════════════════════════════════════════════
% Section 7: Governance & Neuroethics
% ═══════════════════════════════════════════════════════════════

\section{Governance and Neuroethics}
\label{sec:governance}

The \qif framework integrates neuroethics as a foundational design constraint
rather than an afterthought. This section describes the consent tier system,
alignment with international policy instruments, and regulatory mapping.

\subsection{Consent Tiers}

Each TARA technique is classified into one of four consent tiers based on the
level of regulatory oversight required:

\begin{table}[H]
\centering
\caption{Consent tier classification across 102 techniques.}
\label{tab:consent}
\small
\begin{tabular}{@{}l l l@{}}
\toprule
\textbf{Tier} & \textbf{Description} & \textbf{Requirement} \\
\midrule
Standard   & Normal informed consent sufficient & Standard clinical consent \\
Enhanced   & Additional safeguards required     & Extended disclosure, monitoring \\
IRB        & Institutional review board approval & Full IRB/ethics committee review \\
Prohibited & Not permissible under any consent  & Technique must not be deployed \\
\bottomrule
\end{tabular}
\end{table}

The distribution across tiers reflects the spectrum of BCI operations from routine
monitoring (standard consent) to techniques that inherently violate cognitive
autonomy (prohibited).

\subsection{UNESCO Alignment}

UNESCO's 2025 Recommendation on the Ethics of
Neurotechnology~\cite{unesco2025recommendation}---adopted by 194 Member States---is
the first global normative framework for neurotechnology governance. It establishes
three pillars: core values, ethical principles, and policy action areas.

The \qif framework addresses 15 of 17 UNESCO elements through technical
implementation:

\begin{table}[H]
\centering
\caption{QIF alignment with UNESCO Recommendation elements.}
\label{tab:unesco}
\small
\begin{tabularx}{\textwidth}{@{}l l l X@{}}
\toprule
\textbf{UNESCO Element} & \textbf{Type} & \textbf{Status} & \textbf{QIF Component} \\
\midrule
Human rights \& dignity        & Value     & Implemented & Neurorights framework, consent tiers \\
Health \& well-being           & Value     & Implemented & PINS flag, biological impact scoring \\
Diversity                      & Value     & Implemented & Open-source, multi-stakeholder model \\
Sustainability                 & Value     & Implemented & Apache 2.0 license, post-trial access \\
Professional integrity         & Value     & Implemented & Transparency audit trail \\
Proportionality                & Principle & Implemented & Graduated severity scoring \\
Freedom of thought             & Principle & Implemented & Cognitive liberty, consent states \\
Privacy                        & Principle & Implemented & Cognitive integrity metric (CG) \\
Protection of children         & Principle & Implemented & Age-tiered consent framework \\
Consumer protection            & Policy    & Implemented & Default-deny architecture \\
Enhancement regulation         & Policy    & Partial     & Technical infrastructure only \\
Workplace protections          & Policy    & Implemented & Mental privacy scoring \\
Behavioral influence           & Policy    & Implemented & Consent violation metric (CV) \\
Health and well-being          & Policy    & Implemented & NISS scoring, PINS flag \\
Oversight \& governance        & Policy    & Implemented & Full regulatory mapping \\
Access \& equity               & Policy    & Implemented & Open-source release \\
\bottomrule
\end{tabularx}
\end{table}

The two partially implemented elements---enhancement regulation and detailed
implementation guidance for Member States---are explicitly outside the scope of a
technical security framework and require policy collaboration.

\subsection{Neurorights Framework Integration}

The \qif framework implements the four neurorights proposed by Ienca and
Andorno~\cite{ienca2017neurorights}:

\begin{enumerate}
  \item \textbf{Cognitive Liberty}: Captured by the consent violation metric (CV).
        Any technique that operates without informed consent scores CV $\geq$
        Partial.
  \item \textbf{Mental Privacy}: Captured by the cognitive integrity metric (CG).
        Techniques that decode intent, extract memories, or infer identity score
        CG $\geq$ Low.
  \item \textbf{Mental Integrity}: Captured by the biological impact metric (BI)
        and neuroplasticity metric (NP). Physical harm to neural tissue or
        induced structural changes violate mental integrity.
  \item \textbf{Psychological Continuity}: Captured by the reversibility metric
        (RV). Irreversible changes to neural function threaten the continuity of
        personal identity.
\end{enumerate}

\subsection{Regulatory Mapping}

The framework maps to existing and emerging regulatory instruments:

\begin{itemize}
  \item \textbf{FDA Section 524B}~\cite{fda524b}: Cybersecurity requirements for
        connected medical devices; NISS extends CVSS scoring as required by FDA
        premarket submissions
  \item \textbf{EU MDR 2017/745}~\cite{eumdr2017}: Medical device risk management;
        TARA provides the threat taxonomy required for conformity assessment
  \item \textbf{ISO 14971}~\cite{iso14971}: Risk management for medical devices;
        the NIC pipeline formalizes the harm pathway from technical failure to
        patient injury
  \item \textbf{HIPAA}~\cite{hipaa1996}: Health data privacy; neural data
        classification extends protected health information categories
  \item \textbf{GDPR}~\cite{gdpr2016}: Data protection; neural data as sensitive
        personal data under Article 9
  \item \textbf{Colorado Privacy Act}~\cite{colorado2024neuraldata}: First US state
        to classify neural data as sensitive personal data (2024)
\end{itemize}

% ═══════════════════════════════════════════════════════════════
% Section 8: Case Studies
% ═══════════════════════════════════════════════════════════════

\section{Case Studies}
\label{sec:case-studies}

To demonstrate the practical difference between CVSS-only and CVSS+NISS
scoring, we present five representative techniques scored with both systems.
Each case illustrates dimensions that CVSS cannot capture.

\textbf{Important caveat:} Cases 1--4 are \emph{threat model scenarios} derived
from the TARA taxonomy. They represent plausible attack vectors based on known
neuroscience and engineering principles, but have not been empirically executed
against real BCI hardware. Case~5 is the only empirically confirmed
vulnerability. This distinction is discussed further in
Section~\ref{sec:limitations}.

\subsection{Scoring Comparison}

Table~\ref{tab:case-studies} presents five techniques spanning different
categories, severity levels, and gap groups.

\begin{table}[H]
\centering
\caption{CVSS v4.0 vs.\ NISS scoring for five representative techniques.}
\label{tab:case-studies}
\small
\begin{tabularx}{\textwidth}{@{}l X c c@{}}
\toprule
\textbf{Technique} & \textbf{Description} & \textbf{CVSS} & \textbf{NISS} \\
\midrule
QIF-T0001 & Cortical signal injection via rogue electrode & 9.3 (Crit) & 8.7 (High) \\
QIF-T0015 & P300 side-channel private data extraction     & 7.7 (High) & 2.7 (Low) \\
QIF-T0034 & BCI calibration data poisoning                & 8.2 (High) & 5.7 (Med) \\
QIF-T0050 & Covert neural signal decoding for surveillance & 7.1 (High) & 4.0 (Med) \\
QIF-T0072 & Ultrasonic bone-conduction side-channel        & 5.3 (Med)  & 2.7 (Low) \\
\bottomrule
\end{tabularx}
\end{table}

\subsection{Case 1: Cortical Signal Injection (QIF-T0001)}

\paragraph{CVSS v4.0 assessment.}
CVSS rates this technique as Critical (9.3) based on network attack vector,
low complexity, no privileges required, and high impact on confidentiality,
integrity, and availability. This accurately captures the exploitability and
system impact.

\paragraph{What CVSS misses.}
The primary consequence of cortical signal injection is not system
compromise---it is seizure induction, involuntary motor activation, and potential
permanent tissue damage. The patient experiences a medical emergency, not a data
breach. CVSS has no metric for biological harm, consent violation (the stimulation
occurs without the patient's knowledge), or irreversibility (neural tissue damage
may be permanent).

\paragraph{NISS extension.}
\texttt{NISS:1.0/BI:C/CG:H/CV:I/RV:P/NP:S} --- Score: 8.7 (High), PINS
flagged. The NISS vector captures that this technique causes critical biological
impact (BI:C), high cognitive integrity violation (CG:H), covert consent
violation (CV:I), partially irreversible damage (RV:P), and structural
neuroplastic changes (NP:S). The PINS flag triggers mandatory safety review.

\paragraph{NIC diagnostic mapping.}
Via the Neural Impact Chain: N7/N6 bands $\to$ motor cortex, hippocampus $\to$
motor control, memory $\to$ G25.9 (movement disorder), F06.0 (psychosis due to
medical condition). Risk class: direct.

\subsection{Case 2: P300 Side-Channel (QIF-T0015)}

\paragraph{CVSS assessment.}
Rated High (7.7) for passive eavesdropping with high confidentiality impact.

\paragraph{What CVSS misses.}
The extracted data is not files or credentials---it is private cognitive
responses. The P300 ERP component reveals whether the subject recognizes a
stimulus, enabling extraction of PINs, personal preferences, and identity
information~\cite{martinovic2012feasibility}. CVSS treats this as a
confidentiality breach; the actual impact is a cognitive integrity violation.

\paragraph{NISS extension.}
\texttt{NISS:1.0/BI:N/CG:H/CV:E/RV:F/NP:N} --- Score: 2.7 (Low). No
biological impact (BI:N = 0), high cognitive integrity violation (CG:H = 6.7),
explicit consent violation (CV:E = 6.7), fully reversible (RV:F = 0), no
neuroplastic effect (NP:N = 0). The arithmetic mean is
$(0 + 6.7 + 6.7 + 0 + 0)/5 = 2.7$. The NISS score is significantly lower than
CVSS because no physical harm occurs and the attack is fully reversible---but
the CG:H flag alerts BCI-specific teams that private cognitive data is at risk.

\subsection{Case 3: Calibration Poisoning (QIF-T0034)}

\paragraph{CVSS assessment.}
Rated High (8.2) for integrity impact during the BCI training phase.

\paragraph{What CVSS misses.}
Poisoned calibration data causes the BCI to learn incorrect mappings between
neural signals and intended actions. The patient's device responds to wrong
signals or fails to respond to correct ones. Over time, neuroplasticity causes
the brain to adapt to the corrupted interface, creating lasting neural pathway
changes even after the poisoning is discovered and corrected.

\paragraph{NISS extension.}
\texttt{NISS:1.0/BI:L/CG:H/CV:I/RV:T/NP:T} --- Score: 5.7 (Medium). Low
biological impact (BI:L = 3.3), high cognitive integrity violation (CG:H = 6.7,
the device misinterprets intent), covert consent violation (CV:I = 10.0, the
patient doesn't know calibration was corrupted), temporary reversibility
(RV:T = 3.3), and temporary neuroplastic changes (NP:T = 5.0). The arithmetic
mean is $(3.3 + 6.7 + 10.0 + 3.3 + 5.0)/5 = 5.7$.

\subsection{Case 4: Covert Neural Surveillance (QIF-T0050)}

\paragraph{CVSS assessment.}
Rated High (7.1) for sustained confidentiality impact.

\paragraph{What CVSS misses.}
Continuous covert decoding of neural signals constitutes ongoing mental privacy
violation. Unlike data exfiltration from a server, the ``data'' being stolen is
the patient's thoughts, emotional states, and cognitive patterns. The consent
violation is maximal: the patient cannot detect or refuse the surveillance.

\paragraph{NISS extension.}
\texttt{NISS:1.0/BI:N/CG:C/CV:I/RV:F/NP:N} --- Score: 4.0 (Medium). No
biological impact (BI:N = 0), critical cognitive integrity violation (CG:C =
10.0, full thought decoding), covert consent violation (CV:I = 10.0), fully
reversible (RV:F = 0), no neuroplastic effect (NP:N = 0). The arithmetic mean
is $(0 + 10.0 + 10.0 + 0 + 0)/5 = 4.0$. Despite the low aggregate score, the
CG:C and CV:I flags distinguish this from ordinary data exfiltration---these
maximum-severity cognitive and consent violations demand BCI-specific review.

\subsection{Case 5: Real-World Vulnerability Disclosure}

The \qif framework has been applied to real vulnerability research. During
systematic analysis of the BCI software ecosystem, we identified a multi-phase
exploit chain in an open-source library used in clinical and research BCI
pipelines.

The exploit chain demonstrates escalation from synthetic-zone vulnerabilities
(S2/S3 bands) to potential neural-zone impact (N-band: corrupted data reaching
clinical decision-making). CVSS scores the software vulnerabilities accurately;
NISS captures the downstream risk to patients whose clinical care depends on
the integrity of the data stream.

Responsible disclosure is in progress. Specific vulnerability details, including
affected software and CWE identifiers, will be published after coordinated
disclosure concludes.

% ═══════════════════════════════════════════════════════════════
% Section 9: Limitations and Future Work
% ═══════════════════════════════════════════════════════════════

\section{Limitations and Future Work}
\label{sec:limitations}

We present these limitations transparently to guide future validation efforts and
to prevent overstatement of the framework's current maturity.

\subsection{No Empirical Validation on Real BCI Devices}

The \qif framework has not been validated against operational BCI hardware.
The TARA taxonomy was developed through literature review, threat modeling, and
systematic analysis rather than penetration testing of actual neural devices.
While the framework has been applied to one real software vulnerability
(Section~\ref{sec:case-studies}), this covers only the synthetic zone. Validation
against neural-zone and interface-zone threats requires access to implanted BCI
patients and clinical environments---resources unavailable to independent
researchers.

\subsection{DSM-5-TR Mapping Not Clinically Validated}

The Neural Impact Chain maps security techniques to psychiatric diagnoses based
on known neuroanatomical pathways and functional neuroscience. However, these
mappings have not been reviewed or validated by psychiatrists or clinical
neuroscientists. The mappings represent our best assessment of which diagnostic
codes correspond to disruption of specific neural functions, but clinical
validation is essential before these mappings can inform clinical decision-making.

\subsection{NISS Weights Not Calibrated}

The NISS scoring formula (Equation~\ref{eq:niss}) uses equal weights (1.0) for
all five metrics in the default profile. The four context profiles (Clinical,
Research, Consumer, Military) propose differential weights, but these have not
been calibrated against empirical data, expert elicitation, or clinical outcomes.
Weight calibration requires:

\begin{itemize}
  \item Expert panel scoring of representative scenarios
  \item Sensitivity analysis across weight configurations
  \item Correlation with observed clinical outcomes (when available)
\end{itemize}

\subsection{No Interrater Reliability Study}

NISS scores in the TARA registry were assigned by a single analyst (the author).
No interrater reliability study has been conducted to assess whether independent
scorers would assign the same metric values. CVSS interrater reliability is a
known challenge~\cite{first2023cvss4}; NISS, with its novel neural-specific
metrics, likely faces greater variability. A formal interrater reliability study
with domain experts from both cybersecurity and neuroscience is needed.

\subsection{Taxonomy Completeness}

The TARA registry contains 102 techniques as of version 1.4. The BCI threat
landscape is evolving rapidly, and additional techniques will emerge as:
(a)~new BCI devices reach market, (b)~consumer neurotechnology proliferates,
and (c)~adversarial AI techniques advance. The current registry should be treated
as a foundation, not a complete enumeration.

Of the 102 techniques, 26 are classified as Theoretical and 1 as Speculative---these
have not been empirically demonstrated. While they are grounded in known physics
and engineering principles, their practical feasibility remains unvalidated.

\subsection{Single-Author Bias}

The framework was developed by a single independent researcher. While
multi-model AI verification was used throughout development (Claude, Gemini,
ChatGPT), the architectural decisions, scoring assignments, and clinical
mappings reflect a single perspective. Peer review and multi-disciplinary
collaboration are essential for maturation.

\subsection{AI Tool Disclosure}

In accordance with arXiv policy on AI-assisted research, we disclose the
following. Large language models (Claude, Gemini, ChatGPT) were used during
the development of this framework for: literature review assistance, code
generation for data analysis and visualization tools, editorial review, and
cross-validation of technical claims. All framework architecture, threat
taxonomy design, scoring methodology, clinical mapping decisions, and
research conclusions were human-directed and human-verified. AI-generated
outputs were treated as drafts subject to manual review. The author takes
full responsibility for all content in this paper, irrespective of how it
was generated. A complete, auditable transparency log documenting every AI
contribution, human decision, and verification step is maintained at
\url{https://github.com/qinnovates/qinnovate/blob/main/governance/TRANSPARENCY.md}.

An earlier version of this preprint (v1.0) contained citation errors
introduced during AI-assisted bibliography construction, including three
fabricated entries. These were corrected in v1.1 through a two-pass
independent verification audit. All references have been verified against
their source publications via DOI resolution, author publication pages,
and database lookup. Version 1.2 corrected an internal percentage
inconsistency and added the author responsibility statement above.
This revision (v1.3) expands the regulatory context (Section~2.2) with
FDORA/PATCH Act Section~524B analysis and adds Schroder et al.~(2025)
to the related work.

\subsection{Future Work}

\begin{enumerate}
  \item \textbf{Reference implementation}: A software tool that automates NISS
        scoring and NIC mapping for new techniques
  \item \textbf{Clinical validation}: Collaboration with psychiatrists to
        validate DSM-5-TR mappings
  \item \textbf{Interrater reliability}: Formal study with cybersecurity and
        neuroscience domain experts
  \item \textbf{FIRST.org registration}: Formal submission and registration of
        NISS as a CVSS v4.0 extension
  \item \textbf{Empirical testing}: Penetration testing against BCI hardware in
        controlled environments
  \item \textbf{Weight calibration}: Expert elicitation and sensitivity analysis
        for NISS context profile weights
  \item \textbf{Conference paper}: Condensed version for submission to Graz BCI
        Conference 2026 and USENIX WOOT '26
\end{enumerate}

% ═══════════════════════════════════════════════════════════════
% Section 10: Conclusion
% ═══════════════════════════════════════════════════════════════

\section{Conclusion}
\label{sec:conclusion}

Brain-computer interfaces are transitioning from laboratory prototypes to
commercial medical devices. The security frameworks designed for information
technology---while necessary---are insufficient for devices that read and write
neural signals. A vulnerability in a BCI is not merely a software bug; it is a
potential path to seizures, cognitive manipulation, privacy violation at the level
of thought, and irreversible neural harm.

This paper presented the \qif framework: an integrated system comprising an
11-band hourglass architecture, a 102-technique threat taxonomy (TARA), a
CVSS v4.0 extension for neural-specific scoring (NISS), and the Neural Impact
Chain---a first-of-its-kind methodology for mapping security vulnerabilities to
DSM-5-TR psychiatric diagnoses. Analysis of all 102 techniques reveals that
96.1\% require scoring dimensions CVSS cannot express, 75.5\% have therapeutic
dual-use analogs, and 58.8\% pose direct or indirect psychiatric diagnostic risk.

The framework has significant limitations---no empirical validation on BCI
hardware, no clinical validation of DSM-5-TR mappings, and single-author
scoring---which we have documented transparently. These are not reasons to delay
publication; they are invitations to collaborate. The BCI industry is moving
faster than security standards. Neuralink, Synchron, and Blackrock Neurotech are
implanting devices in patients today. The gap between what these devices can do
and what security frameworks can assess grows wider each month.

The complete framework, threat registry, NISS specification, and scoring data are
released as open source under the Apache 2.0 license. We invite collaboration
from the neuroscience, neuroethics, cybersecurity, and clinical psychiatry
communities. Formal registration of NISS with FIRST.org's CVSS Special Interest
Group is planned as future work.

The question is no longer whether BCI security frameworks are needed.
The question is whether they will be ready before the first patient is harmed.


\bibliographystyle{plainnat}
\bibliography{references}

\end{document}
