\section{Introduction}
\label{sec:intro}

Brain-computer interfaces (BCIs) are no longer confined to research laboratories.
Neuralink has implanted its N1 device in human patients~\cite{musk2019neuralink},
Synchron's Stentrode has demonstrated motor neuroprosthesis via neurointerventional
surgery~\cite{oxley2021stentrode}, and Blackrock Neurotech's Utah array has enabled
high-performance speech neuroprostheses~\cite{willett2023speech}. Investment in
neurotechnology grew 700\% between 2014 and 2021, with the global market projected
to reach \$25 billion by 2030~\cite{unesco2025recommendation}.

These devices read and write neural signals. An attacker who compromises a BCI
does not merely exfiltrate data or disrupt a service---they can induce seizures,
decode private thoughts, corrupt memory consolidation, or cause irreversible
brain damage. Yet no security framework exists that accounts for these risks.

\subsection{The Scoring Gap}

The Common Vulnerability Scoring System (CVSS v4.0)~\cite{first2023cvss4} is the
industry standard for vulnerability assessment. CVSS captures exploitability
characteristics and information system impact, but was designed for IT assets.
It cannot express:

\begin{itemize}[nosep]
  \item \textbf{Biological tissue damage}---seizure induction, neural necrosis,
        involuntary motor activation
  \item \textbf{Cognitive integrity}---thought privacy, perception manipulation,
        identity modification
  \item \textbf{Consent boundaries}---covert neural manipulation vs.\ consented
        therapeutic stimulation
  \item \textbf{Damage reversibility}---IT assets restore from backup; neural
        tissue cannot be rebooted
  \item \textbf{Neuroplastic consequences}---prolonged adversarial stimulation
        causes lasting structural brain changes
\end{itemize}

Our analysis of 102 BCI attack techniques shows that 94.4\% require scoring
dimensions that CVSS cannot express. When a vulnerability can induce a
psychiatric diagnosis, a base score alone is insufficient.

\subsection{Contributions}

We present an integrated framework with four contributions:

\begin{enumerate}[nosep]
  \item \textbf{An 11-band hourglass architecture} (Section~\ref{sec:architecture})
        mapping attack surfaces across neural, interface, and synthetic zones, with a
        natural security chokepoint at the electrode-tissue boundary.

  \item \textbf{A threat taxonomy} (Section~\ref{sec:taxonomy}) of 102 techniques
        across 15 tactics and 8 domains, inspired by the structural methodology of
        MITRE ATT\&CK\textsuperscript{\textregistered} and tailored to the BCI domain.

  \item \textbf{A neural impact scoring extension} (Section~\ref{sec:scoring}) for
        CVSS v4.0, adding five neural-specific metrics and designed to conform with
        FIRST.org's official extension mechanism~\cite{first2023cvss4userguide}.

  \item \textbf{A vulnerability-to-diagnosis pipeline} (Section~\ref{sec:impact-chain})
        mapping security vulnerabilities to DSM-5-TR psychiatric
        diagnoses~\cite{apa2022dsm5tr} through a six-stage chain.
\end{enumerate}

We validate the framework through two vulnerability disclosures against real
BCI-adjacent systems (Section~\ref{sec:cases}). The complete framework is
released as open source.
