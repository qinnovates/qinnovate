\section{Conclusion}
\label{sec:conclusion}

Brain-computer interfaces are transitioning from laboratory prototypes to
commercial medical devices. A vulnerability in a BCI is not merely a software
bug---it is a potential path to seizures, cognitive manipulation, thought-level
privacy violation, and irreversible neural harm.

We presented an integrated framework comprising an 11-band architecture, a
102-technique threat taxonomy, a CVSS v4.0 extension with five neural-specific
metrics, and a first-of-its-kind pipeline mapping security vulnerabilities to
DSM-5-TR psychiatric diagnoses. Analysis reveals that 96.1\% of techniques
require scoring dimensions CVSS cannot express, 75.5\% have therapeutic
dual-use analogs, and 58.8\% pose direct or indirect psychiatric diagnostic
risk. We validated the framework through two vulnerability disclosures against
real BCI-adjacent systems.

The framework has significant limitations---documented transparently in
Section~\ref{sec:limitations}. These are invitations to collaborate, not reasons
to delay. The BCI industry is moving faster than security standards. The question
is not whether neural security frameworks are needed. The question is whether
they will be ready before the first patient is harmed.
