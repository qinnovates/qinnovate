\section{Related Work}
\label{sec:related}

Research at the intersection of cybersecurity and neurotechnology has progressed
through foundational framing, empirical demonstration, and emerging policy response.

\paragraph{Foundational Neurosecurity.}
Denning, Matsuoka, and Kohno~\cite{kohno2009neurosecurity}
coined ``neurosecurity'' by analyzing attack surfaces in implantable
neurostimulators, identifying wireless reprogramming, battery depletion, and signal
injection as threat categories. Bonaci et al.~\cite{bonaci2015appstores} extended
this with ``App Stores for the Brain,'' examining privacy threats from third-party
BCI applications.

\paragraph{Empirical Attacks.}
Martinovic et al.~\cite{martinovic2012feasibility} demonstrated at USENIX Security
2012 that consumer EEG headsets could be exploited via P300 event-related potentials
to extract PINs and personal information through subliminal visual stimuli. This
remains the most influential empirical BCI side-channel attack.
Meng et al.~\cite{meng2023adversarial} established an adversarial robustness
benchmark for EEG-based BCIs, evaluating multiple defense approaches across
neural network architectures and datasets. Halperin et al.~\cite{halperin2008pacemakers}
established that implanted medical device attacks are practical by demonstrating
wireless attacks against pacemakers.

\paragraph{Recent Frameworks.}
Schroder et al.~\cite{schroder2025cyberrisks} published the most recent analysis
(2025), developing a threat model for network-based attacks on next-generation BCIs
and identifying associated regulatory changes. Ienca and Haselager~\cite{ienca2016hacking} analyzed BCI security through
informational and physical integrity, proposing that BCIs require both cybersecurity
and neuroethical safeguards. Camara et al.~\cite{camara2015security} and Rushanan
et al.~\cite{rushanan2014sok} surveyed security in implantable medical devices
broadly, covering pacemakers and neurostimulators.

\paragraph{Neuroethics and Policy.}
Ienca and Andorno~\cite{ienca2017neurorights} proposed four neurorights (cognitive
liberty, mental privacy, mental integrity, psychological continuity). UNESCO's
2025 Recommendation~\cite{unesco2025recommendation}, adopted by 194 Member States,
is the first global normative framework for neurotechnology governance.

\paragraph{Gap.}
Prior work addresses BCI security, neuroethics, and vulnerability scoring
separately. No existing framework integrates architecture, technique-level
taxonomy, CVSS-compatible scoring, and clinical impact mapping into a single
system. The vulnerability-to-diagnosis pipeline has no precedent in either
cybersecurity or neuroethics literature.
