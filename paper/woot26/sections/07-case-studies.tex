\section{Case Studies}
\label{sec:cases}

We validate the framework through two channels: scoring comparisons that
demonstrate the gap between CVSS-only and extended scoring, and real
vulnerability disclosures against BCI-adjacent systems.

\textbf{Caveat:} Cases 1--4 are threat model scenarios derived from the taxonomy.
They represent plausible attacks based on known neuroscience and engineering but
have not been executed against real BCI hardware. Case~5 involves an empirically
confirmed vulnerability.

\subsection{Scoring Comparison}

\begin{table}[h]
\centering
\caption{CVSS v4.0 vs.\ neural-extended scoring.}
\label{tab:cases}
\small
\begin{tabular}{@{}p{3.8cm}cc@{}}
\toprule
\textbf{Technique} & \textbf{CVSS} & \textbf{Extension} \\
\midrule
Cortical signal injection  & 9.3 (Crit) & 8.7 (High) \\
P300 side-channel~\cite{martinovic2012feasibility} & 7.7 (High) & 4.7 (Med) \\
Calibration poisoning      & 8.2 (High) & 5.3 (Med) \\
Covert neural surveillance & 7.1 (High) & 6.0 (Med) \\
Ultrasonic bone-conduction & 5.3 (Med)  & 2.7 (Low) \\
\bottomrule
\end{tabular}
\end{table}

The key observation: CVSS and the extension do not always agree on severity
ranking. Cortical signal injection scores high on both because it is both
exploitable \emph{and} biologically devastating. P300 side-channel scores high
on CVSS (confidentiality breach) but medium on the extension (no physical
harm---the attack is passive). This distinction matters for clinical triage.

\subsection{Case 5: BCI Streaming Library Vulnerability}

During systematic analysis of the BCI software ecosystem, we identified a
multi-phase exploit chain in a widely-used open-source library for neural data
streaming. The library is deployed in clinical and research BCI pipelines
across multiple institutions.

The exploit chain demonstrates escalation from synthetic-zone vulnerabilities
(S2/S3 bands) to potential neural-zone impact: corrupted data reaching clinical
decision-making systems. CVSS scores the software vulnerabilities accurately;
the neural extension captures downstream risk to patients whose care depends on
data stream integrity.

Responsible disclosure was initiated prior to this submission. The vulnerability
was reported to maintainers on February 11, 2026. Specific details---including
affected software, CWE identifiers, and proof-of-concept---will be published
after coordinated disclosure concludes.

A second vulnerability---a 9-year-old flaw in a widely-deployed audio codec used
in neural signal processing pipelines---was filed through a national CERT
coordination center. CVE assignment is pending.
