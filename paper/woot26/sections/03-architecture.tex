\section{The Hourglass Architecture}
\label{sec:architecture}

The proposed architecture maps the full BCI attack surface using an 11-band
hourglass model organized into three zones. The model derives from two
independent design traditions: the OSI networking reference
model~\cite{zimmermann1980osi}, which stratifies communication systems into
functional layers, and functional neuroanatomy~\cite{kandel2021neuroscience},
which organizes the nervous system by structure and physiological role. The
hourglass shape borrows from the Internet protocol
architecture~\cite{deering1998hourglass}, where IP serves as a narrow waist
through which all traffic must pass.

\subsection{Design Rationale}

A BCI system spans from cortical computation to radio transmission. The neural
interface---the electrode-tissue boundary---forms a natural chokepoint analogous
to IP: all signals must cross this boundary. Above it, attack surfaces expand
through the complexity of neural tissue. Below it, they expand through electronic
and wireless systems. The resulting shape is an asymmetric hourglass.

\subsection{Band Definitions}

The model uses a 7-1-3 structure: seven neural bands (N7--N1) corresponding to
the canonical CNS hierarchy~\cite{kandel2021neuroscience}, one interface band
(I0), and three synthetic bands (S1--S3).

\begin{table}[h]
\centering
\caption{Hourglass architecture: 11 bands across three zones.}
\label{tab:bands}
\small
\begin{tabular}{@{}llll@{}}
\toprule
\textbf{Band} & \textbf{Name} & \textbf{Zone} & \textbf{Attack Surface} \\
\midrule
N7 & Neocortex       & Neural & Executive function, language, movement \\
N6 & Limbic System   & Neural & Emotion, memory, interoception \\
N5 & Basal Ganglia   & Neural & Motor selection, reward, habit \\
N4 & Diencephalon    & Neural & Sensory gating, consciousness relay \\
N3 & Cerebellum      & Neural & Motor coordination, timing \\
N2 & Brainstem       & Neural & Vital functions, arousal, reflexes \\
N1 & Spinal Cord     & Neural & Reflexes, peripheral relay \\
\midrule
I0 & Neural Interface & Interface & Electrode-tissue boundary \\
\midrule
S1 & Analog          & Synthetic & Amplification, ADC, near-field EM \\
S2 & Digital         & Synthetic & Decoding, BLE/WiFi, telemetry \\
S3 & Radio           & Synthetic & RF, directed energy, app layer \\
\bottomrule
\end{tabular}
\end{table}

\subsection{Security Properties}

The architecture provides three properties: (1)~\emph{completeness}---every BCI
component maps to exactly one band; (2)~\emph{chokepoint identification}---I0
is the natural monitoring point for all bio-digital signal transit; and
(3)~\emph{threat locality}---each technique in the taxonomy is annotated with
affected bands, enabling defenders to understand the spatial extent of an attack.

Ascending the neural hierarchy, attacks produce qualitatively different harms:
N1--N2 attacks threaten vital functions (respiratory arrest, cardiac
dysregulation); N5--N7 attacks threaten cognition, identity, and autonomy. This
\emph{determinacy gradient} has direct security implications: lower-band attacks
are more predictable and immediately dangerous, while higher-band attacks are
less predictable but potentially more insidious.
