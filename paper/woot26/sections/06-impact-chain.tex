\section{From Vulnerability to Diagnosis}
\label{sec:impact-chain}

We introduce a six-stage pipeline for mapping security vulnerabilities to
clinical psychiatric diagnoses. To our knowledge, this is the first systematic
methodology connecting cybersecurity severity to DSM-5-TR diagnostic
codes~\cite{apa2022dsm5tr}.

\subsection{Pipeline}

Each technique is traced through six stages:
(1)~\textbf{Technique}: the attack;
(2)~\textbf{Band}: which hourglass band(s) it affects;
(3)~\textbf{Neural structure}: the anatomy at risk;
(4)~\textbf{Cognitive function}: what that structure supports;
(5)~\textbf{Neural impact score}: particularly the BI, CG, and NP metrics;
(6)~\textbf{DSM-5-TR code}: the psychiatric diagnosis associated with disruption
of that function.

The bridge from scoring metrics to diagnostic clusters is driven by which metric
dominates: BI-driven techniques map to motor/neurocognitive disorders; CG-driven
to cognitive/psychotic clusters; CV-driven to mood/trauma disorders; NP/RV-driven
to persistent personality changes.

\subsection{Coverage}

All 102 techniques have been mapped:

\begin{table}[h]
\centering
\caption{Impact chain mapping results.}
\label{tab:nic-stats}
\small
\begin{tabular}{@{}lr@{}}
\toprule
\textbf{Metric} & \textbf{Value} \\
\midrule
Techniques mapped           & 102 / 102 \\
Unique DSM-5-TR codes       & 15 \\
Diagnostic clusters         & 5 \\
Direct diagnostic risk      & 51 (50.0\%) \\
Indirect diagnostic risk    & 9 (8.8\%) \\
No diagnostic risk          & 42 (41.2\%) \\
\bottomrule
\end{tabular}
\end{table}

The five clusters are: Non-diagnostic~(42), Mood/Trauma~(21, e.g., F43.10 PTSD,
F32.9 MDD), Cognitive/Psychotic~(16, e.g., F06.0 psychosis), Motor/Neurocognitive~(16,
e.g., G25.9 movement disorder), and Persistent/Personality~(7, e.g., F07.0
personality change).

\subsection{Example: Cortical Signal Injection}

Tracing one technique through the full chain:

\begin{enumerate}[nosep]
  \item \textbf{Technique}: Cortical signal injection
  \item \textbf{Band}: N7 (Neocortex), N6 (Limbic)
  \item \textbf{Structure}: Motor cortex (M1), prefrontal cortex, hippocampus
  \item \textbf{Function}: Motor control, executive function, memory
  \item \textbf{Score}: BI:C/CG:H/CV:I/RV:P/NP:S $\to$ 8.7 (High), PINS flagged
  \item \textbf{DSM-5-TR}: G25.9 (movement disorder), F06.0 (psychosis due to
        medical condition), F07.0 (personality change)
  \item \textbf{Risk}: Direct---stimulation itself triggers seizures, involuntary
        movement, perception distortion
\end{enumerate}

A CVSS score of 9.3 tells a security team this vulnerability is critical. The
impact chain tells a clinical team it can cause a movement disorder, psychosis,
and personality change. Both are needed.
