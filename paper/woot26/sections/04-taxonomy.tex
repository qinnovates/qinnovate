\section{Threat Taxonomy}
\label{sec:taxonomy}

We catalog 102 BCI attack techniques with a structure inspired by MITRE
ATT\&CK~\cite{mitre2024attack}, independently developed to cover neural,
cognitive, and physical safety domains. Each technique is simultaneously an
attack vector, an ethical risk, and---where applicable---a therapeutic application.

\paragraph{Why not ATT\&CK directly?}
MITRE ATT\&CK is designed for enterprise IT. BCI threats differ fundamentally:
(1)~the target is biological tissue, not an information system---attacks can
cause seizures or cognitive coercion; (2)~the same technique is often
simultaneously an attack and a therapy (e.g., deep brain stimulation); and
(3)~the attack lifecycle crosses the bio-digital boundary that ATT\&CK's
purely digital model does not address.

\subsection{Structure}

The taxonomy organizes 102 techniques into 15 tactics across 8 domains spanning
the full attack lifecycle: Neural~(N), BCI System~(B), Protocol~(P), Data~(D),
Cognitive~(C), Countermeasure~(M), Evasion~(E), and Sensor~(S).

\subsection{Evidence and Severity}

Each technique carries an evidence status: Confirmed~(19), Demonstrated~(33),
Emerging~(22), Theoretical~(26), Plausible~(1), Speculative~(1). CVSS v4.0 base
severity is heavily weighted toward high and critical: 29 critical (28.4\%), 54
high (52.9\%), 16 medium (15.7\%), 3 low (2.9\%).

\subsection{Dual-Use Mapping}

A distinctive feature is systematic dual-use mapping: every technique is assessed
for therapeutic analogs. The same mechanisms that enable attacks---electromagnetic
stimulation, signal decoding, neuromodulation---underlie established therapies
(DBS~\cite{lozano2019dbs}, TMS~\cite{krauss2021tms}, neurofeedback).

\begin{table}[h]
\centering
\caption{Dual-use classification of 102 techniques.}
\label{tab:dualuse}
\small
\begin{tabular}{@{}llrr@{}}
\toprule
\textbf{Class} & \textbf{Definition} & \textbf{Count} & \textbf{\%} \\
\midrule
Confirmed    & Published clinical use        & 52 & 51.0\% \\
Probable     & Under clinical investigation  & 16 & 15.7\% \\
Possible     & Theoretical therapeutic map   &  9 &  8.8\% \\
Silicon Only & No tissue analog              & 25 & 24.5\% \\
\bottomrule
\end{tabular}
\end{table}

Of the 102 techniques, 77 (75.5\%) have confirmed or probable therapeutic analogs.
The boundary between attack and therapy is not mechanism---it is consent, dosage,
and oversight.

\subsection{Representative Techniques}

\begin{table}[h]
\centering
\caption{Five representative techniques.}
\label{tab:representative}
\small
\begin{tabular}{@{}lllp{4.5cm}@{}}
\toprule
\textbf{Sev.} & \textbf{Status} & \textbf{Dual-Use} & \textbf{Description} \\
\midrule
Critical & Confirmed & Confirmed & Cortical signal injection via rogue electrode stimulation \\
High & Demonstrated & Confirmed & P300 side-channel extraction of private information~\cite{martinovic2012feasibility} \\
High & Emerging & Probable & Calibration data poisoning during BCI training sessions \\
Medium & Confirmed & Confirmed & Ultrasonic side-channel via bone conduction \\
High & Demonstrated & Confirmed & WiFi CSI passive body sensing for respiratory inference \\
\bottomrule
\end{tabular}
\end{table}
