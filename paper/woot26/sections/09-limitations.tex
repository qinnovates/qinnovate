\section{Limitations and Future Work}
\label{sec:limitations}

We present these limitations transparently.

\paragraph{No empirical validation on BCI hardware.}
The taxonomy was developed through literature review and threat modeling, not
penetration testing of neural devices. Only the synthetic zone has been
validated against real software (Section~\ref{sec:cases}).

\paragraph{DSM-5-TR mapping not clinically validated.}
The impact chain mappings have not been reviewed by psychiatrists or clinical
neuroscientists. They represent our best assessment based on neuroanatomical
pathways and functional neuroscience.

\paragraph{Scoring weights not calibrated.}
The extension uses equal weights for all five metrics. Context profiles
(clinical, research, consumer, military) propose differential weights but
lack empirical calibration.

\paragraph{No interrater reliability.}
Scores were assigned by a single analyst. A formal interrater reliability study
with domain experts from both cybersecurity and neuroscience is needed.

\paragraph{Single-author bias.}
The framework was developed by a single independent researcher. Multi-model AI
verification was used throughout development; architectural decisions and scoring
assignments reflect a single perspective. Peer review is essential for maturation.

\paragraph{AI Tool Disclosure.}
In accordance with arXiv policy on AI-assisted research, we disclose the
following. Large language models (Claude, Gemini, ChatGPT) were used during
development for: literature review assistance, code generation for data
analysis and visualization, editorial review, and cross-validation of
technical claims. All framework architecture, threat taxonomy design, scoring
methodology, clinical mapping decisions, and research conclusions were
human-directed and human-verified. AI-generated outputs were treated as drafts
subject to manual review. The author takes full responsibility for all content
in this paper, irrespective of how it was generated. A complete transparency
log is maintained at
\url{https://github.com/qinnovates/qinnovate/blob/main/governance/TRANSPARENCY.md}.

\paragraph{Future work.}
(1)~Clinical validation with psychiatrists; (2)~interrater reliability study;
(3)~formal registration of the scoring extension with FIRST.org;
(4)~penetration testing against BCI hardware; (5)~weight calibration via expert
elicitation.
