\section{Ethics}
\label{sec:ethics}

\paragraph{Beneficence.}
This work presents the first integrated security framework for brain-computer
interfaces, enabling manufacturers, regulators, and researchers to assess
neural-specific risks. The threat taxonomy could theoretically inform attackers;
however, all 102 techniques are derived from published literature or established
therapeutic protocols. No novel attack code is disclosed. 75.5\% of catalogued
techniques have confirmed or probable therapeutic analogs---the knowledge is
already in clinical use.

\paragraph{Respect for persons.}
No human subjects were involved. No BCI devices were tested on patients. All
vulnerability research was conducted on software systems, not devices in clinical
use.

\paragraph{Justice.}
The complete framework is released under the Apache 2.0 license. It is designed
to protect vulnerable populations---BCI patients with paralysis, locked-in
syndrome, and epilepsy---who cannot opt out of their devices.

\paragraph{Responsible disclosure.}
\emph{Vulnerability 1} (BCI data streaming library): Disclosed to maintainers
on February 11, 2026, prior to this submission. We are awaiting establishment of
a private disclosure channel. No exploit code has been published. Full
proof-of-concept is withheld pending vendor patch.
\emph{Vulnerability 2} (audio codec): Filed through a national CERT coordination
center. CVE assignment is pending.

Both disclosures follow coordinated vulnerability disclosure best practices. This
paper does not contain sufficient detail to reproduce either exploit.
