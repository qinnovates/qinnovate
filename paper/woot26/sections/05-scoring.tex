\section{Neural Impact Scoring}
\label{sec:scoring}

We propose a CVSS v4.0 extension designed to conform with FIRST.org's official
extension mechanism (User Guide
\S3.11)~\cite{first2023cvss4userguide}. The extension adds five metrics
capturing dimensions CVSS cannot express.

\subsection{Gap Analysis}

Mapping all 102 techniques to CVSS v4.0 base vectors reveals three gap groups:

\begin{table}[h]
\centering
\caption{CVSS v4.0 gap analysis.}
\label{tab:gap}
\small
\begin{tabular}{@{}clr@{}}
\toprule
\textbf{Group} & \textbf{Gap Description} & \textbf{Count} \\
\midrule
1 & CVSS captures most impact; extension adds nuance & 12 \\
2 & CVSS captures exploitability but misses half & 28 \\
3 & CVSS fundamentally cannot express primary impact & 58 \\
\midrule
  & \textbf{Techniques needing extension} & \textbf{98 (96.1\%)} \\
\bottomrule
\end{tabular}
\end{table}

\subsection{Extension Metrics}

Five metrics, each orthogonal to CVSS base metrics:

\paragraph{Biological Impact (BI).} Direct harm to neural tissue or
physiological function. Values: None~(0.0), Low~(3.3, temporary discomfort),
High~(6.7, seizure induction, involuntary motor activation), Critical~(10.0,
life-threatening or permanently disabling).

\paragraph{Cognitive Integrity (CG).} Impact on thought processes, perception,
memory, or identity. Values: None~(0.0), Low~(3.3, partial intent exposure),
High~(6.7, full thought decoding or perception manipulation), Critical~(10.0,
cognitive coercion or identity modification).

\paragraph{Consent Violation (CV).} Degree of informed consent violation.
Values: None~(0.0), Partial~(3.3, scope exceeded but subject aware),
Explicit~(6.7, direct violation, detectable), Implicit~(10.0, covert
manipulation the patient cannot detect).

\paragraph{Reversibility (RV).} Whether damage can be undone. Values:
Full~(0.0, effects cease with attack), Temporary~(3.3, hours to days),
Partial~(6.7, some permanent), Irreversible~(10.0, permanent neural
destruction).

\paragraph{Neuroplasticity (NP).} Whether the attack exploits or induces
neuroplastic changes. Values: None~(0.0), Temporary~(5.0, short-term synaptic),
Structural~(10.0, long-term pathway changes).

\subsection{PINS Flag}

The Potential Impact to Neural Safety (PINS) flag triggers when
$\text{BI} \geq \text{High}$ or $\text{RV} = \text{Irreversible}$, mandating
immediate safety review. 31 of 102 techniques (30.4\%) are PINS-flagged.

\subsection{Dual-Vector Architecture}

The extension vector rides alongside CVSS:

{\small\texttt{CVSS:4.0/AV:N/AC:L/AT:N/PR:N/UI:N/VC:H/VI:H/VA:H/SC:N/SI:N/SA:N}}

{\small\texttt{NISS:1.0/BI:H/CG:C/CV:I/RV:P/NP:S}}

Security teams triage using familiar CVSS scores while BCI-specific teams see
the neural dimensions that determine whether a vulnerability is a software bug
or a patient safety emergency.
