\documentclass[letterpaper,twocolumn,10pt]{article}
\usepackage{usenix2019_v3}

\usepackage{amsmath}
\usepackage{booktabs}
\usepackage{tabularx}
\usepackage{graphicx}
\usepackage{enumitem}

%-------------------------------------------------------------------------------
\begin{document}
%-------------------------------------------------------------------------------

\date{}

\title{\Large \bf From Attack Surface to Psychiatric Diagnosis:\\
  A Security Framework for Brain-Computer Interfaces}

% DOUBLE-BLIND: Authors removed for review
\author{
{\rm Anonymous}
}

\maketitle

%-------------------------------------------------------------------------------
\begin{abstract}
%-------------------------------------------------------------------------------
Brain-computer interfaces (BCIs) are transitioning from experimental tools to
commercially deployed medical devices. Yet no security framework accounts for
the unique risks of devices that read and write neural signals. The Common
Vulnerability Scoring System (CVSS v4.0) cannot express biological tissue
damage, cognitive integrity violations, or consent boundaries---dimensions
critical to neural device security.

We present an integrated framework with four contributions:
(1)~an 11-band hourglass architecture mapping attack surfaces from neocortex
to wireless radio;
(2)~a threat taxonomy of 102 techniques across 15 tactics and 8 domains;
(3)~a CVSS v4.0 extension adding five neural-specific metrics; and
(4)~a methodology mapping security vulnerabilities to DSM-5-TR psychiatric
diagnoses.

Analysis reveals that 94.4\% of catalogued techniques require neural-specific
metrics that CVSS cannot express, and 51 techniques pose direct psychiatric
diagnostic risk. We validate the framework through two vulnerability
disclosures against real BCI-adjacent systems. The complete framework is
released as open source.
\end{abstract}

% ═══════════════════════════════════════════════════════════════
% Section 1: Introduction
% ═══════════════════════════════════════════════════════════════

\section{Introduction}
\label{sec:introduction}

Brain-computer interfaces (BCIs) are no longer confined to research laboratories.
Neuralink has implanted its N1 device in human patients~\cite{musk2019neuralink},
Synchron's Stentrode has demonstrated motor neuroprosthesis via neurointerventional
surgery~\cite{oxley2021stentrode}, and Blackrock Neurotech's Utah array has enabled
high-performance speech neuroprostheses~\cite{willett2023speech}. Paradromics is
advancing toward high-bandwidth cortical interfaces for paralyzed patients.
Investment in neurotechnology grew 700\% between 2014 and 2021, with the global
market projected to reach \$25 billion by 2030~\cite{unesco2025recommendation}.

Despite this rapid commercialization, no security standard exists specifically for
neural devices. Section 3305 of the Food and Drug Omnibus Reform Act of 2022
(FDORA, Pub.~L. 117-328) added Section 524B to the FD\&C Act~\cite{fda524b},
mandating cybersecurity documentation---including threat modeling, software bills
of materials, and vulnerability disclosure---in all premarket submissions for
connected medical devices. Since October 2023, the FDA enforces a
Refuse-to-Accept policy for non-compliant submissions. However, Section 524B
does not specify which threat taxonomy or scoring system to use, and the
referenced standards (CVSS, IEC 62443, AAMI TIR57, ISO 14971) contain no
provisions for the unique risks of devices that read and write neural signals.
The EU Medical Device Regulation~\cite{eumdr2017} similarly lacks neural-specific
security requirements.

\subsection{The Scoring Gap}

The Common Vulnerability Scoring System (CVSS v4.0)~\cite{first2023cvss4} is the
industry standard for assessing vulnerability severity. CVSS excels at capturing
exploitability characteristics---attack vector, complexity, privileges required,
user interaction---and information system impact across confidentiality, integrity,
and availability. However, CVSS was designed for information technology assets.
It has no concept of:

\begin{itemize}
  \item \textbf{Biological tissue damage}---seizure induction, neural tissue
        necrosis, involuntary motor activation
  \item \textbf{Cognitive integrity}---thought privacy, perception manipulation,
        identity modification
  \item \textbf{Consent boundaries}---the distinction between operating within
        consented parameters and covert neural manipulation
  \item \textbf{Damage reversibility}---IT assets can be restored from backup;
        neural tissue cannot be rebooted
  \item \textbf{Neuroplastic consequences}---prolonged adversarial stimulation can
        cause lasting structural changes to the brain
\end{itemize}

When a vulnerability in a BCI device can induce seizures, decode private thoughts,
or cause irreversible brain damage, a CVSS base score alone is insufficient.
The gap is not theoretical: our analysis of 102 BCI attack techniques shows that
96.1\% require scoring dimensions that CVSS cannot express.

\subsection{Contributions}

This paper presents an integrated security framework with four contributions:

\begin{enumerate}
  \item \textbf{The Hourglass Architecture} (Section~\ref{sec:hourglass}): An
        11-band model mapping attack surfaces across three zones---neural (7~bands,
        from neocortex to spinal cord), interface (1~band, the electrode-tissue
        boundary), and synthetic (3~bands, from analog electronics to wireless
        radio). The hourglass shape reflects a natural security chokepoint at the
        neural interface.

  \item \textbf{TARA Threat Taxonomy} (Section~\ref{sec:tara}): A registry of
        102~techniques across 15~tactics and 8~domains, each classified by
        evidence status, severity, and dual-use therapeutic potential. TARA is
        an independent taxonomy inspired by the structural methodology of MITRE
        ATT\&CK\textsuperscript{\textregistered}, tailored to the BCI domain.

  \item \textbf{NISS Scoring System} (Section~\ref{sec:niss}): The Neural Impact
        Scoring System---a CVSS v4.0 extension designed to conform with
        FIRST.org's official extension mechanism (User Guide
        \S3.11)~\cite{first2023cvss4userguide}, adding five neural-specific
        metrics. Formal registration with FIRST.org is planned as future work.

  \item \textbf{The Neural Impact Chain} (Section~\ref{sec:nic}): A methodology
        mapping security vulnerabilities to psychiatric diagnoses through a
        six-stage pipeline: technique $\to$ hourglass band $\to$ neural
        structure $\to$ cognitive function $\to$ NISS score $\to$ DSM-5-TR
        diagnostic code~\cite{apa2022dsm5tr}.
\end{enumerate}

Sections~\ref{sec:governance}--\ref{sec:case-studies} address governance alignment
with international neuroethics frameworks and present comparative case studies.
Section~\ref{sec:limitations} discusses limitations and validation gaps honestly.
The complete framework, threat registry, and scoring system are released as open
source under the Apache 2.0 license.

\section{Related Work}
\label{sec:related}

Research at the intersection of cybersecurity and neurotechnology has progressed
through foundational framing, empirical demonstration, and emerging policy response.

\paragraph{Foundational Neurosecurity.}
Denning, Matsuoka, and Kohno~\cite{kohno2009neurosecurity}
coined ``neurosecurity'' by analyzing attack surfaces in implantable
neurostimulators, identifying wireless reprogramming, battery depletion, and signal
injection as threat categories. Bonaci et al.~\cite{bonaci2015appstores} extended
this with ``App Stores for the Brain,'' examining privacy threats from third-party
BCI applications.

\paragraph{Empirical Attacks.}
Martinovic et al.~\cite{martinovic2012feasibility} demonstrated at USENIX Security
2012 that consumer EEG headsets could be exploited via P300 event-related potentials
to extract PINs and personal information through subliminal visual stimuli. This
remains the most influential empirical BCI side-channel attack.
Meng et al.~\cite{meng2023adversarial} established an adversarial robustness
benchmark for EEG-based BCIs, evaluating multiple defense approaches across
neural network architectures and datasets. Halperin et al.~\cite{halperin2008pacemakers}
established that implanted medical device attacks are practical by demonstrating
wireless attacks against pacemakers.

\paragraph{Recent Frameworks.}
Schroder et al.~\cite{schroder2025cyberrisks} published the most recent analysis
(2025), developing a threat model for network-based attacks on next-generation BCIs
and identifying associated regulatory changes. Ienca and Haselager~\cite{ienca2016hacking} analyzed BCI security through
informational and physical integrity, proposing that BCIs require both cybersecurity
and neuroethical safeguards. Camara et al.~\cite{camara2015security} and Rushanan
et al.~\cite{rushanan2014sok} surveyed security in implantable medical devices
broadly, covering pacemakers and neurostimulators.

\paragraph{Neuroethics and Policy.}
Ienca and Andorno~\cite{ienca2017neurorights} proposed four neurorights (cognitive
liberty, mental privacy, mental integrity, psychological continuity). UNESCO's
2025 Recommendation~\cite{unesco2025recommendation}, adopted by 194 Member States,
is the first global normative framework for neurotechnology governance.

\paragraph{Gap.}
Prior work addresses BCI security, neuroethics, and vulnerability scoring
separately. No existing framework integrates architecture, technique-level
taxonomy, CVSS-compatible scoring, and clinical impact mapping into a single
system. The vulnerability-to-diagnosis pipeline has no precedent in either
cybersecurity or neuroethics literature.

\section{The Hourglass Architecture}
\label{sec:architecture}

The proposed architecture maps the full BCI attack surface using an 11-band
hourglass model organized into three zones. The model derives from two
independent design traditions: the OSI networking reference
model~\cite{zimmermann1980osi}, which stratifies communication systems into
functional layers, and functional neuroanatomy~\cite{kandel2021neuroscience},
which organizes the nervous system by structure and physiological role. The
hourglass shape borrows from the Internet protocol
architecture~\cite{deering1998hourglass}, where IP serves as a narrow waist
through which all traffic must pass.

\subsection{Design Rationale}

A BCI system spans from cortical computation to radio transmission. The neural
interface---the electrode-tissue boundary---forms a natural chokepoint analogous
to IP: all signals must cross this boundary. Above it, attack surfaces expand
through the complexity of neural tissue. Below it, they expand through electronic
and wireless systems. The resulting shape is an asymmetric hourglass.

\subsection{Band Definitions}

The model uses a 7-1-3 structure: seven neural bands (N7--N1) corresponding to
the canonical CNS hierarchy~\cite{kandel2021neuroscience}, one interface band
(I0), and three synthetic bands (S1--S3).

\begin{table}[h]
\centering
\caption{Hourglass architecture: 11 bands across three zones.}
\label{tab:bands}
\small
\begin{tabular}{@{}llll@{}}
\toprule
\textbf{Band} & \textbf{Name} & \textbf{Zone} & \textbf{Attack Surface} \\
\midrule
N7 & Neocortex       & Neural & Executive function, language, movement \\
N6 & Limbic System   & Neural & Emotion, memory, interoception \\
N5 & Basal Ganglia   & Neural & Motor selection, reward, habit \\
N4 & Diencephalon    & Neural & Sensory gating, consciousness relay \\
N3 & Cerebellum      & Neural & Motor coordination, timing \\
N2 & Brainstem       & Neural & Vital functions, arousal, reflexes \\
N1 & Spinal Cord     & Neural & Reflexes, peripheral relay \\
\midrule
I0 & Neural Interface & Interface & Electrode-tissue boundary \\
\midrule
S1 & Analog          & Synthetic & Amplification, ADC, near-field EM \\
S2 & Digital         & Synthetic & Decoding, BLE/WiFi, telemetry \\
S3 & Radio           & Synthetic & RF, directed energy, app layer \\
\bottomrule
\end{tabular}
\end{table}

\subsection{Security Properties}

The architecture provides three properties: (1)~\emph{completeness}---every BCI
component maps to exactly one band; (2)~\emph{chokepoint identification}---I0
is the natural monitoring point for all bio-digital signal transit; and
(3)~\emph{threat locality}---each technique in the taxonomy is annotated with
affected bands, enabling defenders to understand the spatial extent of an attack.

Ascending the neural hierarchy, attacks produce qualitatively different harms:
N1--N2 attacks threaten vital functions (respiratory arrest, cardiac
dysregulation); N5--N7 attacks threaten cognition, identity, and autonomy. This
\emph{determinacy gradient} has direct security implications: lower-band attacks
are more predictable and immediately dangerous, while higher-band attacks are
less predictable but potentially more insidious.

\section{Threat Taxonomy}
\label{sec:taxonomy}

We catalog 102 BCI attack techniques with a structure inspired by MITRE
ATT\&CK~\cite{mitre2024attack}, independently developed to cover neural,
cognitive, and physical safety domains. Each technique is simultaneously an
attack vector, an ethical risk, and---where applicable---a therapeutic application.

\paragraph{Why not ATT\&CK directly?}
MITRE ATT\&CK is designed for enterprise IT. BCI threats differ fundamentally:
(1)~the target is biological tissue, not an information system---attacks can
cause seizures or cognitive coercion; (2)~the same technique is often
simultaneously an attack and a therapy (e.g., deep brain stimulation); and
(3)~the attack lifecycle crosses the bio-digital boundary that ATT\&CK's
purely digital model does not address.

\subsection{Structure}

The taxonomy organizes 102 techniques into 15 tactics across 8 domains spanning
the full attack lifecycle: Neural~(N), BCI System~(B), Protocol~(P), Data~(D),
Cognitive~(C), Countermeasure~(M), Evasion~(E), and Sensor~(S).

\subsection{Evidence and Severity}

Each technique carries an evidence status: Confirmed~(19), Demonstrated~(33),
Emerging~(22), Theoretical~(26), Plausible~(1), Speculative~(1). CVSS v4.0 base
severity is heavily weighted toward high and critical: 29 critical (28.4\%), 54
high (52.9\%), 16 medium (15.7\%), 3 low (2.9\%).

\subsection{Dual-Use Mapping}

A distinctive feature is systematic dual-use mapping: every technique is assessed
for therapeutic analogs. The same mechanisms that enable attacks---electromagnetic
stimulation, signal decoding, neuromodulation---underlie established therapies
(DBS~\cite{lozano2019dbs}, TMS~\cite{krauss2021tms}, neurofeedback).

\begin{table}[h]
\centering
\caption{Dual-use classification of 102 techniques.}
\label{tab:dualuse}
\small
\begin{tabular}{@{}llrr@{}}
\toprule
\textbf{Class} & \textbf{Definition} & \textbf{Count} & \textbf{\%} \\
\midrule
Confirmed    & Published clinical use        & 52 & 51.0\% \\
Probable     & Under clinical investigation  & 16 & 15.7\% \\
Possible     & Theoretical therapeutic map   &  9 &  8.8\% \\
Silicon Only & No tissue analog              & 25 & 24.5\% \\
\bottomrule
\end{tabular}
\end{table}

Of the 102 techniques, 77 (75.5\%) have confirmed or probable therapeutic analogs.
The boundary between attack and therapy is not mechanism---it is consent, dosage,
and oversight.

\subsection{Representative Techniques}

\begin{table}[h]
\centering
\caption{Five representative techniques.}
\label{tab:representative}
\small
\begin{tabular}{@{}lllp{4.5cm}@{}}
\toprule
\textbf{Sev.} & \textbf{Status} & \textbf{Dual-Use} & \textbf{Description} \\
\midrule
Critical & Confirmed & Confirmed & Cortical signal injection via rogue electrode stimulation \\
High & Demonstrated & Confirmed & P300 side-channel extraction of private information~\cite{martinovic2012feasibility} \\
High & Emerging & Probable & Calibration data poisoning during BCI training sessions \\
Medium & Confirmed & Confirmed & Ultrasonic side-channel via bone conduction \\
High & Demonstrated & Confirmed & WiFi CSI passive body sensing for respiratory inference \\
\bottomrule
\end{tabular}
\end{table}

\section{Neural Impact Scoring}
\label{sec:scoring}

We propose a CVSS v4.0 extension designed to conform with FIRST.org's official
extension mechanism (User Guide
\S3.11)~\cite{first2023cvss4userguide}. The extension adds five metrics
capturing dimensions CVSS cannot express.

\subsection{Gap Analysis}

Mapping all 102 techniques to CVSS v4.0 base vectors reveals three gap groups:

\begin{table}[h]
\centering
\caption{CVSS v4.0 gap analysis.}
\label{tab:gap}
\small
\begin{tabular}{@{}clr@{}}
\toprule
\textbf{Group} & \textbf{Gap Description} & \textbf{Count} \\
\midrule
1 & CVSS captures most impact; extension adds nuance & 12 \\
2 & CVSS captures exploitability but misses half & 28 \\
3 & CVSS fundamentally cannot express primary impact & 58 \\
\midrule
  & \textbf{Techniques needing extension} & \textbf{98 (96.1\%)} \\
\bottomrule
\end{tabular}
\end{table}

\subsection{Extension Metrics}

Five metrics, each orthogonal to CVSS base metrics:

\paragraph{Biological Impact (BI).} Direct harm to neural tissue or
physiological function. Values: None~(0.0), Low~(3.3, temporary discomfort),
High~(6.7, seizure induction, involuntary motor activation), Critical~(10.0,
life-threatening or permanently disabling).

\paragraph{Cognitive Integrity (CG).} Impact on thought processes, perception,
memory, or identity. Values: None~(0.0), Low~(3.3, partial intent exposure),
High~(6.7, full thought decoding or perception manipulation), Critical~(10.0,
cognitive coercion or identity modification).

\paragraph{Consent Violation (CV).} Degree of informed consent violation.
Values: None~(0.0), Partial~(3.3, scope exceeded but subject aware),
Explicit~(6.7, direct violation, detectable), Implicit~(10.0, covert
manipulation the patient cannot detect).

\paragraph{Reversibility (RV).} Whether damage can be undone. Values:
Full~(0.0, effects cease with attack), Temporary~(3.3, hours to days),
Partial~(6.7, some permanent), Irreversible~(10.0, permanent neural
destruction).

\paragraph{Neuroplasticity (NP).} Whether the attack exploits or induces
neuroplastic changes. Values: None~(0.0), Temporary~(5.0, short-term synaptic),
Structural~(10.0, long-term pathway changes).

\subsection{PINS Flag}

The Potential Impact to Neural Safety (PINS) flag triggers when
$\text{BI} \geq \text{High}$ or $\text{RV} = \text{Irreversible}$, mandating
immediate safety review. 31 of 102 techniques (30.4\%) are PINS-flagged.

\subsection{Dual-Vector Architecture}

The extension vector rides alongside CVSS:

{\small\texttt{CVSS:4.0/AV:N/AC:L/AT:N/PR:N/UI:N/VC:H/VI:H/VA:H/SC:N/SI:N/SA:N}}

{\small\texttt{NISS:1.0/BI:H/CG:C/CV:I/RV:P/NP:S}}

Security teams triage using familiar CVSS scores while BCI-specific teams see
the neural dimensions that determine whether a vulnerability is a software bug
or a patient safety emergency.

\section{From Vulnerability to Diagnosis}
\label{sec:impact-chain}

We introduce a six-stage pipeline for mapping security vulnerabilities to
clinical psychiatric diagnoses. To our knowledge, this is the first systematic
methodology connecting cybersecurity severity to DSM-5-TR diagnostic
codes~\cite{apa2022dsm5tr}.

\subsection{Pipeline}

Each technique is traced through six stages:
(1)~\textbf{Technique}: the attack;
(2)~\textbf{Band}: which hourglass band(s) it affects;
(3)~\textbf{Neural structure}: the anatomy at risk;
(4)~\textbf{Cognitive function}: what that structure supports;
(5)~\textbf{Neural impact score}: particularly the BI, CG, and NP metrics;
(6)~\textbf{DSM-5-TR code}: the psychiatric diagnosis associated with disruption
of that function.

The bridge from scoring metrics to diagnostic clusters is driven by which metric
dominates: BI-driven techniques map to motor/neurocognitive disorders; CG-driven
to cognitive/psychotic clusters; CV-driven to mood/trauma disorders; NP/RV-driven
to persistent personality changes.

\subsection{Coverage}

All 102 techniques have been mapped:

\begin{table}[h]
\centering
\caption{Impact chain mapping results.}
\label{tab:nic-stats}
\small
\begin{tabular}{@{}lr@{}}
\toprule
\textbf{Metric} & \textbf{Value} \\
\midrule
Techniques mapped           & 102 / 102 \\
Unique DSM-5-TR codes       & 15 \\
Diagnostic clusters         & 5 \\
Direct diagnostic risk      & 51 (50.0\%) \\
Indirect diagnostic risk    & 9 (8.8\%) \\
No diagnostic risk          & 42 (41.2\%) \\
\bottomrule
\end{tabular}
\end{table}

The five clusters are: Non-diagnostic~(42), Mood/Trauma~(21, e.g., F43.10 PTSD,
F32.9 MDD), Cognitive/Psychotic~(16, e.g., F06.0 psychosis), Motor/Neurocognitive~(16,
e.g., G25.9 movement disorder), and Persistent/Personality~(7, e.g., F07.0
personality change).

\subsection{Example: Cortical Signal Injection}

Tracing one technique through the full chain:

\begin{enumerate}[nosep]
  \item \textbf{Technique}: Cortical signal injection
  \item \textbf{Band}: N7 (Neocortex), N6 (Limbic)
  \item \textbf{Structure}: Motor cortex (M1), prefrontal cortex, hippocampus
  \item \textbf{Function}: Motor control, executive function, memory
  \item \textbf{Score}: BI:C/CG:H/CV:I/RV:P/NP:S $\to$ 8.7 (High), PINS flagged
  \item \textbf{DSM-5-TR}: G25.9 (movement disorder), F06.0 (psychosis due to
        medical condition), F07.0 (personality change)
  \item \textbf{Risk}: Direct---stimulation itself triggers seizures, involuntary
        movement, perception distortion
\end{enumerate}

A CVSS score of 9.3 tells a security team this vulnerability is critical. The
impact chain tells a clinical team it can cause a movement disorder, psychosis,
and personality change. Both are needed.

\section{Case Studies}
\label{sec:cases}

We validate the framework through two channels: scoring comparisons that
demonstrate the gap between CVSS-only and extended scoring, and real
vulnerability disclosures against BCI-adjacent systems.

\textbf{Caveat:} Cases 1--4 are threat model scenarios derived from the taxonomy.
They represent plausible attacks based on known neuroscience and engineering but
have not been executed against real BCI hardware. Case~5 involves an empirically
confirmed vulnerability.

\subsection{Scoring Comparison}

\begin{table}[h]
\centering
\caption{CVSS v4.0 vs.\ neural-extended scoring.}
\label{tab:cases}
\small
\begin{tabular}{@{}p{3.8cm}cc@{}}
\toprule
\textbf{Technique} & \textbf{CVSS} & \textbf{Extension} \\
\midrule
Cortical signal injection  & 9.3 (Crit) & 8.7 (High) \\
P300 side-channel~\cite{martinovic2012feasibility} & 7.7 (High) & 4.7 (Med) \\
Calibration poisoning      & 8.2 (High) & 5.3 (Med) \\
Covert neural surveillance & 7.1 (High) & 6.0 (Med) \\
Ultrasonic bone-conduction & 5.3 (Med)  & 2.7 (Low) \\
\bottomrule
\end{tabular}
\end{table}

The key observation: CVSS and the extension do not always agree on severity
ranking. Cortical signal injection scores high on both because it is both
exploitable \emph{and} biologically devastating. P300 side-channel scores high
on CVSS (confidentiality breach) but medium on the extension (no physical
harm---the attack is passive). This distinction matters for clinical triage.

\subsection{Case 5: BCI Streaming Library Vulnerability}

During systematic analysis of the BCI software ecosystem, we identified a
multi-phase exploit chain in a widely-used open-source library for neural data
streaming. The library is deployed in clinical and research BCI pipelines
across multiple institutions.

The exploit chain demonstrates escalation from synthetic-zone vulnerabilities
(S2/S3 bands) to potential neural-zone impact: corrupted data reaching clinical
decision-making systems. CVSS scores the software vulnerabilities accurately;
the neural extension captures downstream risk to patients whose care depends on
data stream integrity.

Responsible disclosure was initiated prior to this submission. The vulnerability
was reported to maintainers on February 11, 2026. Specific details---including
affected software, CWE identifiers, and proof-of-concept---will be published
after coordinated disclosure concludes.

A second vulnerability---a 9-year-old flaw in a widely-deployed audio codec used
in neural signal processing pipelines---was filed through a national CERT
coordination center. CVE assignment is pending.

\section{Ethics}
\label{sec:ethics}

\paragraph{Beneficence.}
This work presents the first integrated security framework for brain-computer
interfaces, enabling manufacturers, regulators, and researchers to assess
neural-specific risks. The threat taxonomy could theoretically inform attackers;
however, all 102 techniques are derived from published literature or established
therapeutic protocols. No novel attack code is disclosed. 75.5\% of catalogued
techniques have confirmed or probable therapeutic analogs---the knowledge is
already in clinical use.

\paragraph{Respect for persons.}
No human subjects were involved. No BCI devices were tested on patients. All
vulnerability research was conducted on software systems, not devices in clinical
use.

\paragraph{Justice.}
The complete framework is released under the Apache 2.0 license. It is designed
to protect vulnerable populations---BCI patients with paralysis, locked-in
syndrome, and epilepsy---who cannot opt out of their devices.

\paragraph{Responsible disclosure.}
\emph{Vulnerability 1} (BCI data streaming library): Disclosed to maintainers
on February 11, 2026, prior to this submission. We are awaiting establishment of
a private disclosure channel. No exploit code has been published. Full
proof-of-concept is withheld pending vendor patch.
\emph{Vulnerability 2} (audio codec): Filed through a national CERT coordination
center. CVE assignment is pending.

Both disclosures follow coordinated vulnerability disclosure best practices. This
paper does not contain sufficient detail to reproduce either exploit.

% ═══════════════════════════════════════════════════════════════
% Section 9: Limitations and Future Work
% ═══════════════════════════════════════════════════════════════

\section{Limitations and Future Work}
\label{sec:limitations}

We present these limitations transparently to guide future validation efforts and
to prevent overstatement of the framework's current maturity.

\subsection{No Empirical Validation on Real BCI Devices}

The \qif framework has not been validated against operational BCI hardware.
The TARA taxonomy was developed through literature review, threat modeling, and
systematic analysis rather than penetration testing of actual neural devices.
While the framework has been applied to one real software vulnerability
(Section~\ref{sec:case-studies}), this covers only the synthetic zone. Validation
against neural-zone and interface-zone threats requires access to implanted BCI
patients and clinical environments---resources unavailable to independent
researchers.

\subsection{DSM-5-TR Mapping Not Clinically Validated}

The Neural Impact Chain maps security techniques to psychiatric diagnoses based
on known neuroanatomical pathways and functional neuroscience. However, these
mappings have not been reviewed or validated by psychiatrists or clinical
neuroscientists. The mappings represent our best assessment of which diagnostic
codes correspond to disruption of specific neural functions, but clinical
validation is essential before these mappings can inform clinical decision-making.

\subsection{NISS Weights Not Calibrated}

The NISS scoring formula (Equation~\ref{eq:niss}) uses equal weights (1.0) for
all five metrics in the default profile. The four context profiles (Clinical,
Research, Consumer, Military) propose differential weights, but these have not
been calibrated against empirical data, expert elicitation, or clinical outcomes.
Weight calibration requires:

\begin{itemize}
  \item Expert panel scoring of representative scenarios
  \item Sensitivity analysis across weight configurations
  \item Correlation with observed clinical outcomes (when available)
\end{itemize}

\subsection{No Interrater Reliability Study}

NISS scores in the TARA registry were assigned by a single analyst (the author).
No interrater reliability study has been conducted to assess whether independent
scorers would assign the same metric values. CVSS interrater reliability is a
known challenge~\cite{first2023cvss4}; NISS, with its novel neural-specific
metrics, likely faces greater variability. A formal interrater reliability study
with domain experts from both cybersecurity and neuroscience is needed.

\subsection{Taxonomy Completeness}

The TARA registry contains 102 techniques as of version 1.4. The BCI threat
landscape is evolving rapidly, and additional techniques will emerge as:
(a)~new BCI devices reach market, (b)~consumer neurotechnology proliferates,
and (c)~adversarial AI techniques advance. The current registry should be treated
as a foundation, not a complete enumeration.

Of the 102 techniques, 26 are classified as Theoretical and 1 as Speculative---these
have not been empirically demonstrated. While they are grounded in known physics
and engineering principles, their practical feasibility remains unvalidated.

\subsection{Single-Author Bias}

The framework was developed by a single independent researcher. While
multi-model AI verification was used throughout development (Claude, Gemini,
ChatGPT), the architectural decisions, scoring assignments, and clinical
mappings reflect a single perspective. Peer review and multi-disciplinary
collaboration are essential for maturation.

\subsection{AI Tool Disclosure}

In accordance with arXiv policy on AI-assisted research, we disclose the
following. Large language models (Claude, Gemini, ChatGPT) were used during
the development of this framework for: literature review assistance, code
generation for data analysis and visualization tools, editorial review, and
cross-validation of technical claims. All framework architecture, threat
taxonomy design, scoring methodology, clinical mapping decisions, and
research conclusions were human-directed and human-verified. AI-generated
outputs were treated as drafts subject to manual review. The author takes
full responsibility for all content in this paper, irrespective of how it
was generated. A complete, auditable transparency log documenting every AI
contribution, human decision, and verification step is maintained at
\url{https://github.com/qinnovates/qinnovate/blob/main/governance/TRANSPARENCY.md}.

An earlier version of this preprint (v1.0) contained citation errors
introduced during AI-assisted bibliography construction, including three
fabricated entries. These were corrected in v1.1 through a two-pass
independent verification audit. All references have been verified against
their source publications via DOI resolution, author publication pages,
and database lookup. Version 1.2 corrected an internal percentage
inconsistency and added the author responsibility statement above.
This revision (v1.3) expands the regulatory context (Section~2.2) with
FDORA/PATCH Act Section~524B analysis and adds Schroder et al.~(2025)
to the related work.

\subsection{Future Work}

\begin{enumerate}
  \item \textbf{Reference implementation}: A software tool that automates NISS
        scoring and NIC mapping for new techniques
  \item \textbf{Clinical validation}: Collaboration with psychiatrists to
        validate DSM-5-TR mappings
  \item \textbf{Interrater reliability}: Formal study with cybersecurity and
        neuroscience domain experts
  \item \textbf{FIRST.org registration}: Formal submission and registration of
        NISS as a CVSS v4.0 extension
  \item \textbf{Empirical testing}: Penetration testing against BCI hardware in
        controlled environments
  \item \textbf{Weight calibration}: Expert elicitation and sensitivity analysis
        for NISS context profile weights
  \item \textbf{Conference paper}: Condensed version for submission to Graz BCI
        Conference 2026 and USENIX WOOT '26
\end{enumerate}

% ═══════════════════════════════════════════════════════════════
% Section 10: Conclusion
% ═══════════════════════════════════════════════════════════════

\section{Conclusion}
\label{sec:conclusion}

Brain-computer interfaces are transitioning from laboratory prototypes to
commercial medical devices. The security frameworks designed for information
technology---while necessary---are insufficient for devices that read and write
neural signals. A vulnerability in a BCI is not merely a software bug; it is a
potential path to seizures, cognitive manipulation, privacy violation at the level
of thought, and irreversible neural harm.

This paper presented the \qif framework: an integrated system comprising an
11-band hourglass architecture, a 102-technique threat taxonomy (TARA), a
CVSS v4.0 extension for neural-specific scoring (NISS), and the Neural Impact
Chain---a first-of-its-kind methodology for mapping security vulnerabilities to
DSM-5-TR psychiatric diagnoses. Analysis of all 102 techniques reveals that
96.1\% require scoring dimensions CVSS cannot express, 75.5\% have therapeutic
dual-use analogs, and 58.8\% pose direct or indirect psychiatric diagnostic risk.

The framework has significant limitations---no empirical validation on BCI
hardware, no clinical validation of DSM-5-TR mappings, and single-author
scoring---which we have documented transparently. These are not reasons to delay
publication; they are invitations to collaborate. The BCI industry is moving
faster than security standards. Neuralink, Synchron, and Blackrock Neurotech are
implanting devices in patients today. The gap between what these devices can do
and what security frameworks can assess grows wider each month.

The complete framework, threat registry, NISS specification, and scoring data are
released as open source under the Apache 2.0 license. We invite collaboration
from the neuroscience, neuroethics, cybersecurity, and clinical psychiatry
communities. Formal registration of NISS with FIRST.org's CVSS Special Interest
Group is planned as future work.

The question is no longer whether BCI security frameworks are needed.
The question is whether they will be ready before the first patient is harmed.


%-------------------------------------------------------------------------------
\section*{Availability}
%-------------------------------------------------------------------------------

The complete framework, threat registry, and scoring system are released
as open source under the Apache 2.0 license.%
\footnote{Repository URL withheld for double-blind review.}
An interactive threat registry browser and scoring calculator are available
at the project website.

%-------------------------------------------------------------------------------
\bibliographystyle{plain}
\bibliography{references}

\end{document}
