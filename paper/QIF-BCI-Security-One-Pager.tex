\documentclass[11pt]{article}
\usepackage[margin=0.75in]{geometry}
\usepackage{enumitem}
\usepackage{hyperref}
\usepackage{booktabs}
\usepackage{tabularx}
\usepackage{xcolor}

\definecolor{accent}{HTML}{2563EB}
\hypersetup{colorlinks=true, urlcolor=accent, linkcolor=accent}

\pagestyle{empty}
\setlength{\parindent}{0pt}
\setlength{\parskip}{6pt}

\begin{document}

\begin{center}
{\Large\bfseries QIF Framework for Brain-Computer Interface Security}\\[4pt]
{\normalsize Kevin L.\ Qi $\cdot$ Qinnovate $\cdot$ \href{https://qinnovate.com}{qinnovate.com}}\\[2pt]
{\small Preprint: \href{https://doi.org/10.5281/zenodo.18640106}{DOI: 10.5281/zenodo.18640106} $\cdot$ Apache 2.0 $\cdot$ February 2026}
\end{center}

\vspace{-4pt}
\hrule
\vspace{8pt}

\textbf{Problem.} Brain-computer interfaces are shipping in humans (Neuralink, Synchron, Blackrock Neurotech, Paradromics). No security framework addresses biological tissue damage, cognitive integrity violations, or consent boundaries. CVSS~v4.0 --- the industry standard --- cannot express these dimensions.

\textbf{Solution.} An integrated framework with four components:

\begin{enumerate}[leftmargin=1.5em, topsep=2pt, itemsep=2pt]
\item \textbf{11-Band Hourglass Architecture} --- Maps every attack surface from neocortex (N7) to wireless radio (S3) across neural, interface, and synthetic zones. The interface band (I0) is the natural chokepoint where all signals cross the bio-digital boundary.

\item \textbf{TARA: Threat Taxonomy} --- 102 techniques across 15 tactics and 8 domains. Each classified by evidence status, severity, and dual-use therapeutic potential. 77 techniques (75.5\%) have confirmed or probable therapeutic analogs.

\item \textbf{NISS: Neural Impact Scoring} --- A CVSS~v4.0 extension adding 5 metrics: Biological Impact (BI), Cognitive Integrity (CG), Consent Violation (CV), Reversibility (RV), and Neuroplasticity (NP). 96.1\% of techniques require these extension metrics.

\item \textbf{Neural Impact Chain} --- Maps security vulnerabilities to DSM-5-TR psychiatric diagnoses through a 6-stage pipeline. 15 unique diagnostic codes across 5 clusters; 51 techniques pose direct diagnostic risk.
\end{enumerate}

\textbf{UNESCO Alignment.} The framework addresses \textbf{15 of 17 elements} in the UNESCO Recommendation on the Ethics of Neurotechnology (2025). The two gaps --- enhancement regulation and Member State implementation guidance --- require policy expertise beyond engineering.

\vspace{4pt}

\begin{tabularx}{\textwidth}{@{}l X r@{}}
\toprule
\textbf{Metric} & \textbf{Description} & \textbf{Value} \\
\midrule
Attack techniques catalogued & Across 15 tactics, 8 domains & 102 \\
Techniques needing NISS & CVSS alone insufficient & 96.1\% \\
Dual-use therapeutic analogs & Attack = therapy (with consent) & 75.5\% \\
DSM-5-TR diagnostic codes mapped & Via Neural Impact Chain & 15 \\
Direct diagnostic risk & Techniques with clinical consequence & 51 \\
PINS-flagged techniques & Mandatory safety review & 31 \\
UNESCO elements addressed & Out of 17 total & 15 \\
Vulnerability disclosures filed & Real BCI-adjacent systems & 2 \\
\bottomrule
\end{tabularx}

\vspace{6pt}

\textbf{Key Insight.} The same mechanisms that enable attacks --- electromagnetic stimulation, signal decoding, neuromodulation --- are the mechanisms underlying established therapies (DBS, TMS, neurofeedback). Security and safety are the same problem. The boundary between attack and therapy is not mechanism --- it is consent, dosage, and oversight.

\textbf{Open Source.} Framework, threat registry, scoring system, and governance:\\
\href{https://github.com/qinnovates/qinnovate}{github.com/qinnovates/qinnovate} $\cdot$ Apache 2.0

\end{document}
