% ═══════════════════════════════════════════════════════════════
% Section 4: TARA Threat Taxonomy
% ═══════════════════════════════════════════════════════════════

\section{TARA: Threat Taxonomy}
\label{sec:tara}

The Therapeutic Applications \& Risk Assessment (TARA) registry catalogs BCI
attack techniques with a structure inspired by MITRE
ATT\&CK~\cite{mitre2024attack}, independently developed to cover neural,
cognitive, and physical safety domains. Each
technique is simultaneously an attack vector, an ethical risk, and---where
applicable---a therapeutic application.

\paragraph{Why not ATT\&CK directly?}
MITRE ATT\&CK is designed for enterprise IT and mobile environments where
adversary objectives center on data exfiltration, lateral movement, and
persistence. BCI threats differ in three fundamental ways that make direct
adoption inadequate: (1)~the target is biological tissue, not an information
system---attacks can cause seizures, tissue necrosis, or cognitive coercion,
none of which map to ATT\&CK's impact categories; (2)~the same technique is
often simultaneously an attack and a therapy (e.g., deep brain stimulation is
both a Parkinson's treatment and a neural injection vector), requiring a
dual-use classification that ATT\&CK has no mechanism to express; and (3)~the
attack lifecycle must span from radio-frequency transmission through silicon
processing to neural tissue, crossing the bio-digital boundary that ATT\&CK's
purely digital model does not address. TARA adopts ATT\&CK's proven structural
methodology---techniques organized into tactics and domains---while building an
independent taxonomy purpose-built for the BCI threat landscape.

\subsection{Taxonomy Structure}

The TARA registry organizes 102 techniques into 15 tactics across 8 operational
domains. Technique identifiers follow the format \texttt{QIF-TXXX}, where the
numeric suffix provides sequential ordering.

\paragraph{Domains.}
The eight domains span the full attack lifecycle:

\begin{enumerate}
  \item \textbf{Neural (N)} --- Direct interaction with neural tissue
        (scan, injection, manipulation)
  \item \textbf{BCI System (B)} --- System-level intrusion and evasion
  \item \textbf{Protocol (P)} --- Protocol disruption and communication attacks
  \item \textbf{Data (D)} --- Data harvesting and exfiltration
  \item \textbf{Cognitive (C)} --- Cognitive exploitation and imprinting
  \item \textbf{Countermeasure (M)} --- Surveillance and monitoring
  \item \textbf{Evasion (E)} --- Defense evasion and anti-detection
  \item \textbf{Sensor (S)} --- Consumer device side-channel attacks
\end{enumerate}

\paragraph{Tactics.}
The 15 tactics follow a lifecycle structure analogous to ATT\&CK, adapted for BCI operations:
Neural Scan, BCI Intrusion, Neural Injection, Cognitive Imprinting, BCI Evasion,
Data Harvesting, Neural Manipulation, Evasion/Rootkit Deployment, Monitoring/Surveillance,
Protocol Disruption, Cognitive Exploitation, Signal Replay, Signal Harvesting,
Signal Fingerprinting, and Signal Chaining.

\subsection{Evidence Classification}

Each technique carries an evidence status reflecting the maturity of its
documentation:

\begin{table}[H]
\centering
\caption{TARA evidence status classification with technique counts.}
\label{tab:status}
\small
\begin{tabular}{@{}l l r@{}}
\toprule
\textbf{Status} & \textbf{Definition} & \textbf{Count} \\
\midrule
Confirmed    & Documented in real-world use or peer-reviewed literature & 19 \\
Demonstrated & Proven in laboratory or controlled conditions            & 33 \\
Emerging     & Newly identified; limited but growing evidence           & 22 \\
Theoretical  & Plausible based on known physics and engineering          & 26 \\
Plausible    & Possible but with significant uncertainty                & 1 \\
Speculative  & Hypothetical; requires unproven capabilities             & 1 \\
\bottomrule
\end{tabular}
\end{table}

\subsection{Severity Distribution}

CVSS v4.0 base severity ratings for all 102 techniques show a distribution
heavily weighted toward high and critical:

\begin{table}[H]
\centering
\caption{TARA severity distribution (CVSS v4.0 base ratings).}
\label{tab:severity}
\small
\begin{tabular}{@{}l r r@{}}
\toprule
\textbf{Severity} & \textbf{Count} & \textbf{Percentage} \\
\midrule
Critical & 29 & 28.4\% \\
High     & 54 & 52.9\% \\
Medium   & 16 & 15.7\% \\
Low      &  3 &  2.9\% \\
\bottomrule
\end{tabular}
\end{table}

The dominance of high and critical ratings reflects the inherent severity of
attacks against devices that interface directly with the nervous system. Even
techniques with low exploitability can have severe consequences when the target
is neural tissue.

\subsection{Category Breakdown}

The 102 techniques distribute across eight operational categories:

\begin{table}[H]
\centering
\caption{TARA techniques by operational category.}
\label{tab:categories}
\small
\begin{tabular}{@{}l l r@{}}
\toprule
\textbf{ID} & \textbf{Category} & \textbf{Count} \\
\midrule
SE & Signal Eavesdropping    & 20 \\
CI & Cognitive Integrity     & 18 \\
EX & Data Exfiltration       & 17 \\
DM & Data Manipulation       & 15 \\
SI & Signal Injection        & 10 \\
PE & Privilege Escalation    &  8 \\
DS & Denial of Service       &  7 \\
PS & Physical Safety         &  7 \\
\bottomrule
\end{tabular}
\end{table}

\subsection{Dual-Use Mapping}

A distinctive feature of TARA is the systematic dual-use mapping: every attack
technique is assessed for therapeutic analogs. The same physical mechanisms that
enable attacks---electromagnetic stimulation, signal decoding, neuromodulation---are
the mechanisms underlying established therapies such as deep brain stimulation
(DBS)~\cite{lozano2019dbs}, transcranial magnetic stimulation
(TMS)~\cite{hallett2007tms}, and neurofeedback.

\begin{table}[H]
\centering
\caption{Dual-use classification of 102 TARA techniques.}
\label{tab:dualuse}
\small
\begin{tabular}{@{}l l r r@{}}
\toprule
\textbf{Classification} & \textbf{Definition} & \textbf{Count} & \textbf{\%} \\
\midrule
Confirmed    & Published clinical use exists      & 52 & 51.0\% \\
Probable     & Under active clinical investigation & 16 & 15.7\% \\
Possible     & Theoretical therapeutic mapping     &  9 &  8.8\% \\
Silicon Only & No tissue analog; purely digital    & 25 & 24.5\% \\
\bottomrule
\end{tabular}
\end{table}

Of the 102 techniques, 77 (75.5\%) have confirmed, probable, or possible therapeutic analogs.
This finding underscores a fundamental challenge for BCI security: the same
capabilities that must be defended against are often the capabilities that make
BCIs therapeutically valuable.

\subsection{Representative Techniques}

Table~\ref{tab:representative} shows five representative techniques spanning
different categories, severities, and dual-use classifications.

\begin{table}[H]
\centering
\caption{Five representative TARA techniques.}
\label{tab:representative}
\small
\begin{tabularx}{\textwidth}{@{}l l l l X@{}}
\toprule
\textbf{ID} & \textbf{Sev.} & \textbf{Status} & \textbf{Dual-Use} & \textbf{Description} \\
\midrule
T0001 & Critical & Confirmed & Confirmed & Cortical signal injection via rogue electrode stimulation; therapeutic analog: deep brain stimulation \\
T0015 & High & Demonstrated & Confirmed & P300 side-channel extraction of private information; therapeutic analog: P300-based spelling interfaces \\
T0034 & High & Emerging & Probable & Calibration data poisoning during BCI training sessions; therapeutic analog: adaptive neurofeedback \\
T0072 & Medium & Confirmed & Confirmed & Ultrasonic side-channel via bone conduction microphone; therapeutic analog: ABR audiometry \\
T0090 & High & Demonstrated & Confirmed & WiFi CSI body sensing for respiratory and gait inference; therapeutic analog: sleep apnea detection \\
\bottomrule
\end{tabularx}
\end{table}

Figure~\ref{fig:severity-dist} shows the severity distribution across all
techniques, and Figure~\ref{fig:dual-use-breakdown} illustrates the dual-use
breakdown.

\begin{figure}[H]
  \centering
  \begin{minipage}[t]{0.48\textwidth}
    \centering
    \includegraphics[width=\textwidth]{figures/severity-dist.pdf}
    \caption{TARA severity distribution across 102 techniques.}
    \label{fig:severity-dist}
  \end{minipage}
  \hfill
  \begin{minipage}[t]{0.48\textwidth}
    \centering
    \includegraphics[width=\textwidth]{figures/dual-use-breakdown.pdf}
    \caption{Dual-use classification: 75.5\% of techniques have therapeutic analogs.}
    \label{fig:dual-use-breakdown}
  \end{minipage}
\end{figure}
