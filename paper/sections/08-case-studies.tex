% ═══════════════════════════════════════════════════════════════
% Section 8: Case Studies
% ═══════════════════════════════════════════════════════════════

\section{Case Studies}
\label{sec:case-studies}

To demonstrate the practical difference between CVSS-only and CVSS+NISS
scoring, we present five representative techniques scored with both systems.
Each case illustrates dimensions that CVSS cannot capture.

\textbf{Important caveat:} Cases 1--4 are \emph{threat model scenarios} derived
from the TARA taxonomy. They represent plausible attack vectors based on known
neuroscience and engineering principles, but have not been empirically executed
against real BCI hardware. Case~5 is the only empirically confirmed
vulnerability. This distinction is discussed further in
Section~\ref{sec:limitations}.

\subsection{Scoring Comparison}

Table~\ref{tab:case-studies} presents five techniques spanning different
categories, severity levels, and gap groups.

\begin{table}[H]
\centering
\caption{CVSS v4.0 vs.\ NISS scoring for five representative techniques.}
\label{tab:case-studies}
\small
\begin{tabularx}{\textwidth}{@{}l X c c@{}}
\toprule
\textbf{Technique} & \textbf{Description} & \textbf{CVSS} & \textbf{NISS} \\
\midrule
QIF-T0001 & Cortical signal injection via rogue electrode & 9.3 (Crit) & 8.7 (High) \\
QIF-T0015 & P300 side-channel private data extraction     & 7.7 (High) & 2.7 (Low) \\
QIF-T0034 & BCI calibration data poisoning                & 8.2 (High) & 5.7 (Med) \\
QIF-T0050 & Covert neural signal decoding for surveillance & 7.1 (High) & 4.0 (Med) \\
QIF-T0072 & Ultrasonic bone-conduction side-channel        & 5.3 (Med)  & 2.7 (Low) \\
\bottomrule
\end{tabularx}
\end{table}

\subsection{Case 1: Cortical Signal Injection (QIF-T0001)}

\paragraph{CVSS v4.0 assessment.}
CVSS rates this technique as Critical (9.3) based on network attack vector,
low complexity, no privileges required, and high impact on confidentiality,
integrity, and availability. This accurately captures the exploitability and
system impact.

\paragraph{What CVSS misses.}
The primary consequence of cortical signal injection is not system
compromise---it is seizure induction, involuntary motor activation, and potential
permanent tissue damage. The patient experiences a medical emergency, not a data
breach. CVSS has no metric for biological harm, consent violation (the stimulation
occurs without the patient's knowledge), or irreversibility (neural tissue damage
may be permanent).

\paragraph{NISS extension.}
\texttt{NISS:1.0/BI:C/CG:H/CV:I/RV:P/NP:S} --- Score: 8.7 (High), PINS
flagged. The NISS vector captures that this technique causes critical biological
impact (BI:C), high cognitive integrity violation (CG:H), covert consent
violation (CV:I), partially irreversible damage (RV:P), and structural
neuroplastic changes (NP:S). The PINS flag triggers mandatory safety review.

\paragraph{NIC diagnostic mapping.}
Via the Neural Impact Chain: N7/N6 bands $\to$ motor cortex, hippocampus $\to$
motor control, memory $\to$ G25.9 (movement disorder), F06.0 (psychosis due to
medical condition). Risk class: direct.

\subsection{Case 2: P300 Side-Channel (QIF-T0015)}

\paragraph{CVSS assessment.}
Rated High (7.7) for passive eavesdropping with high confidentiality impact.

\paragraph{What CVSS misses.}
The extracted data is not files or credentials---it is private cognitive
responses. The P300 ERP component reveals whether the subject recognizes a
stimulus, enabling extraction of PINs, personal preferences, and identity
information~\cite{martinovic2012feasibility}. CVSS treats this as a
confidentiality breach; the actual impact is a cognitive integrity violation.

\paragraph{NISS extension.}
\texttt{NISS:1.0/BI:N/CG:H/CV:E/RV:F/NP:N} --- Score: 2.7 (Low). No
biological impact (BI:N = 0), high cognitive integrity violation (CG:H = 6.7),
explicit consent violation (CV:E = 6.7), fully reversible (RV:F = 0), no
neuroplastic effect (NP:N = 0). The arithmetic mean is
$(0 + 6.7 + 6.7 + 0 + 0)/5 = 2.7$. The NISS score is significantly lower than
CVSS because no physical harm occurs and the attack is fully reversible---but
the CG:H flag alerts BCI-specific teams that private cognitive data is at risk.

\subsection{Case 3: Calibration Poisoning (QIF-T0034)}

\paragraph{CVSS assessment.}
Rated High (8.2) for integrity impact during the BCI training phase.

\paragraph{What CVSS misses.}
Poisoned calibration data causes the BCI to learn incorrect mappings between
neural signals and intended actions. The patient's device responds to wrong
signals or fails to respond to correct ones. Over time, neuroplasticity causes
the brain to adapt to the corrupted interface, creating lasting neural pathway
changes even after the poisoning is discovered and corrected.

\paragraph{NISS extension.}
\texttt{NISS:1.0/BI:L/CG:H/CV:I/RV:T/NP:T} --- Score: 5.7 (Medium). Low
biological impact (BI:L = 3.3), high cognitive integrity violation (CG:H = 6.7,
the device misinterprets intent), covert consent violation (CV:I = 10.0, the
patient doesn't know calibration was corrupted), temporary reversibility
(RV:T = 3.3), and temporary neuroplastic changes (NP:T = 5.0). The arithmetic
mean is $(3.3 + 6.7 + 10.0 + 3.3 + 5.0)/5 = 5.7$.

\subsection{Case 4: Covert Neural Surveillance (QIF-T0050)}

\paragraph{CVSS assessment.}
Rated High (7.1) for sustained confidentiality impact.

\paragraph{What CVSS misses.}
Continuous covert decoding of neural signals constitutes ongoing mental privacy
violation. Unlike data exfiltration from a server, the ``data'' being stolen is
the patient's thoughts, emotional states, and cognitive patterns. The consent
violation is maximal: the patient cannot detect or refuse the surveillance.

\paragraph{NISS extension.}
\texttt{NISS:1.0/BI:N/CG:C/CV:I/RV:F/NP:N} --- Score: 4.0 (Medium). No
biological impact (BI:N = 0), critical cognitive integrity violation (CG:C =
10.0, full thought decoding), covert consent violation (CV:I = 10.0), fully
reversible (RV:F = 0), no neuroplastic effect (NP:N = 0). The arithmetic mean
is $(0 + 10.0 + 10.0 + 0 + 0)/5 = 4.0$. Despite the low aggregate score, the
CG:C and CV:I flags distinguish this from ordinary data exfiltration---these
maximum-severity cognitive and consent violations demand BCI-specific review.

\subsection{Case 5: Real-World Vulnerability Disclosure}

The \qif framework has been applied to real vulnerability research. During
systematic analysis of the BCI software ecosystem, we identified a multi-phase
exploit chain in an open-source library used in clinical and research BCI
pipelines.

The exploit chain demonstrates escalation from synthetic-zone vulnerabilities
(S2/S3 bands) to potential neural-zone impact (N-band: corrupted data reaching
clinical decision-making). CVSS scores the software vulnerabilities accurately;
NISS captures the downstream risk to patients whose clinical care depends on
the integrity of the data stream.

Responsible disclosure is in progress. Specific vulnerability details, including
affected software and CWE identifiers, will be published after coordinated
disclosure concludes.
