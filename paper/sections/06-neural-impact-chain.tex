% ═══════════════════════════════════════════════════════════════
% Section 6: Neural Impact Chain
% ═══════════════════════════════════════════════════════════════

\section{Neural Impact Chain}
\label{sec:nic}

The Neural Impact Chain (NIC) is a six-stage methodology for mapping security
vulnerabilities to clinical psychiatric diagnoses. To our knowledge, this is the
first systematic pipeline connecting cybersecurity severity scoring to DSM-5-TR
diagnostic codes~\cite{apa2022dsm5tr}. The NIC answers a question no prior
framework has addressed: \emph{if this attack succeeds, what psychiatric condition
could it cause or worsen?}

\subsection{Pipeline Architecture}

The NIC traces each technique through six stages:

\begin{enumerate}
  \item \textbf{Technique}: The TARA attack technique (e.g., cortical signal
        injection)
  \item \textbf{Hourglass Band}: Which band(s) the technique affects (e.g., N7
        Neocortex, N6 Limbic)
  \item \textbf{Neural Structure}: The anatomical structure at risk (e.g.,
        hippocampus, prefrontal cortex, amygdala)
  \item \textbf{Cognitive Function}: The function that structure supports (e.g.,
        memory consolidation, executive control, emotional regulation)
  \item \textbf{NISS Score}: The neural impact score, particularly the BI, CG,
        and NP metrics that correlate with clinical outcomes
  \item \textbf{DSM-5-TR Code}: The ICD-10-CM diagnostic code(s) for the
        psychiatric condition most closely associated with disruption of that
        function
\end{enumerate}

Figure~\ref{fig:nic} illustrates this pipeline.

\begin{figure}[H]
  \centering
  \includegraphics[width=0.85\textwidth]{figures/neural-impact-chain.pdf}
  \caption{The Neural Impact Chain: six-stage pipeline from security technique
  to psychiatric diagnosis. Each arrow represents a mapping grounded in
  neuroanatomy, functional neuroscience, or clinical psychiatry.}
  \label{fig:nic}
\end{figure}

\subsection{NISS-to-DSM Bridge}

The bridge between NISS metrics and DSM-5-TR diagnostic clusters is driven by
which NISS metric dominates the technique's profile:

\begin{itemize}
  \item \textbf{BI-driven} $\to$ Motor/Neurocognitive cluster: Biological
        impact implies tissue damage, leading to movement disorders or
        neurocognitive deficits
  \item \textbf{CG-driven} $\to$ Cognitive/Psychotic cluster: Cognitive
        integrity violations imply perceptual or thought-process disruption,
        leading to psychotic or dissociative symptoms
  \item \textbf{CV-driven} $\to$ Mood/Trauma cluster: Consent violations
        imply autonomy loss, mapping to trauma- and stressor-related disorders
  \item \textbf{NP/RV-driven} $\to$ Persistent/Personality cluster:
        Neuroplasticity exploitation or irreversible damage implies lasting
        personality or behavioral changes
  \item \textbf{No neural impact} $\to$ Non-diagnostic: Silicon-only
        techniques with no direct psychiatric mapping
\end{itemize}

\subsection{Coverage Statistics}

All 102 TARA techniques have been mapped through the NIC pipeline:

\begin{table}[H]
\centering
\caption{Neural Impact Chain mapping results across 102 techniques.}
\label{tab:nic-stats}
\small
\begin{tabular}{@{}l r@{}}
\toprule
\textbf{Metric} & \textbf{Value} \\
\midrule
Techniques mapped                 & 102 / 102 (100\%) \\
Unique DSM-5-TR codes             & 15 \\
Diagnostic clusters               & 5 \\
Direct diagnostic risk             & 51 (50.0\%) \\
Indirect diagnostic risk           & 9 (8.8\%) \\
No diagnostic risk (silicon-only)  & 42 (41.2\%) \\
\bottomrule
\end{tabular}
\end{table}

\subsection{Diagnostic Cluster Distribution}

The five diagnostic clusters and their technique counts:

\begin{table}[H]
\centering
\caption{DSM-5-TR diagnostic cluster distribution.}
\label{tab:clusters}
\small
\begin{tabular}{@{}l r l@{}}
\toprule
\textbf{Cluster} & \textbf{Count} & \textbf{Representative DSM-5-TR Codes} \\
\midrule
Non-diagnostic          & 42 & --- (silicon-only techniques) \\
Mood/Trauma             & 21 & F43.10 (PTSD), F32.9 (MDD), F44.9 (dissociative) \\
Cognitive/Psychotic     & 16 & F06.0 (psychosis due to medical condition), R41.3 (cognitive decline) \\
Motor/Neurocognitive    & 16 & G25.9 (movement disorder), G31.84 (neurocognitive) \\
Persistent/Personality  &  7 & F07.0 (personality change due to medical condition) \\
\bottomrule
\end{tabular}
\end{table}

The Mood/Trauma cluster is the largest diagnostic cluster (21 techniques),
reflecting the prevalence of consent-violation and autonomy-disruption attacks
in the BCI threat landscape. Techniques that covertly manipulate neural signals
without the subject's knowledge or consent map naturally to trauma- and
stressor-related disorders.

\subsection{Risk Classification}

Each technique is classified by diagnostic risk:

\begin{itemize}
  \item \textbf{Direct} (51 techniques, 50.0\%): The attack mechanism can
        directly trigger or worsen the mapped psychiatric condition. Example:
        forced cortical stimulation causing seizures maps to epilepsy-related
        diagnostic codes.
  \item \textbf{Indirect} (9 techniques, 8.8\%): The attack creates downstream
        conditions that may lead to the diagnosis. Example: sustained data
        exfiltration of private thoughts causing anxiety does not directly
        produce the anxiety disorder but creates the conditions for it.
  \item \textbf{None} (42 techniques, 41.2\%): Silicon-only techniques with no
        direct neural interaction and thus no psychiatric diagnostic mapping.
\end{itemize}

\subsection{Example Walkthrough}

Consider technique \textbf{QIF-T0001: Cortical Signal Injection}---direct
injection of adversarial signals via rogue electrode stimulation.

\begin{enumerate}
  \item \textbf{Technique}: QIF-T0001 (cortical signal injection)
  \item \textbf{Band}: N7 (Neocortex), N6 (Limbic System)
  \item \textbf{Structure}: Primary motor cortex (M1), prefrontal cortex (PFC),
        hippocampus
  \item \textbf{Function}: Motor control, executive function, memory
        consolidation
  \item \textbf{NISS}: BI:C / CG:H / CV:I / RV:P / NP:S $\to$ Score: 8.7
        (High), PINS flagged
  \item \textbf{DSM-5-TR}: G25.9 (movement disorder NOS), F06.0 (psychotic
        disorder due to another medical condition), F07.0 (personality change
        due to another medical condition)
  \item \textbf{Risk class}: Direct---the stimulation itself can trigger
        seizures, involuntary movement, and perception distortion
\end{enumerate}

Figure~\ref{fig:dsm5-clusters} shows the distribution of techniques across
diagnostic clusters.

\begin{figure}[H]
  \centering
  \includegraphics[width=0.65\textwidth]{figures/dsm5-clusters.pdf}
  \caption{DSM-5-TR cluster distribution across 102 TARA techniques. 60
  techniques (58.8\%) have direct or indirect diagnostic risk.}
  \label{fig:dsm5-clusters}
\end{figure}
