% ═══════════════════════════════════════════════════════════════
% Section 7: Governance & Neuroethics
% ═══════════════════════════════════════════════════════════════

\section{Governance and Neuroethics}
\label{sec:governance}

The \qif framework integrates neuroethics as a foundational design constraint
rather than an afterthought. This section describes the consent tier system,
alignment with international policy instruments, and regulatory mapping.

\subsection{Consent Tiers}

Each TARA technique is classified into one of four consent tiers based on the
level of regulatory oversight required:

\begin{table}[H]
\centering
\caption{Consent tier classification across 102 techniques.}
\label{tab:consent}
\small
\begin{tabular}{@{}l l l@{}}
\toprule
\textbf{Tier} & \textbf{Description} & \textbf{Requirement} \\
\midrule
Standard   & Normal informed consent sufficient & Standard clinical consent \\
Enhanced   & Additional safeguards required     & Extended disclosure, monitoring \\
IRB        & Institutional review board approval & Full IRB/ethics committee review \\
Prohibited & Not permissible under any consent  & Technique must not be deployed \\
\bottomrule
\end{tabular}
\end{table}

The distribution across tiers reflects the spectrum of BCI operations from routine
monitoring (standard consent) to techniques that inherently violate cognitive
autonomy (prohibited).

\subsection{UNESCO Alignment}

UNESCO's 2025 Recommendation on the Ethics of
Neurotechnology~\cite{unesco2025recommendation}---adopted by 194 Member States---is
the first global normative framework for neurotechnology governance. It establishes
three pillars: core values, ethical principles, and policy action areas.

The \qif framework addresses 15 of 17 UNESCO elements through technical
implementation:

\begin{table}[H]
\centering
\caption{QIF alignment with UNESCO Recommendation elements.}
\label{tab:unesco}
\small
\begin{tabularx}{\textwidth}{@{}l l l X@{}}
\toprule
\textbf{UNESCO Element} & \textbf{Type} & \textbf{Status} & \textbf{QIF Component} \\
\midrule
Human rights \& dignity        & Value     & Implemented & Neurorights framework, consent tiers \\
Health \& well-being           & Value     & Implemented & PINS flag, biological impact scoring \\
Diversity                      & Value     & Implemented & Open-source, multi-stakeholder model \\
Sustainability                 & Value     & Implemented & Apache 2.0 license, post-trial access \\
Professional integrity         & Value     & Implemented & Transparency audit trail \\
Proportionality                & Principle & Implemented & Graduated severity scoring \\
Freedom of thought             & Principle & Implemented & Cognitive liberty, consent states \\
Privacy                        & Principle & Implemented & Cognitive integrity metric (CG) \\
Protection of children         & Principle & Implemented & Age-tiered consent framework \\
Consumer protection            & Policy    & Implemented & Default-deny architecture \\
Enhancement regulation         & Policy    & Partial     & Technical infrastructure only \\
Workplace protections          & Policy    & Implemented & Mental privacy scoring \\
Behavioral influence           & Policy    & Implemented & Consent violation metric (CV) \\
Health and well-being          & Policy    & Implemented & NISS scoring, PINS flag \\
Oversight \& governance        & Policy    & Implemented & Full regulatory mapping \\
Access \& equity               & Policy    & Implemented & Open-source release \\
\bottomrule
\end{tabularx}
\end{table}

The two partially implemented elements---enhancement regulation and detailed
implementation guidance for Member States---are explicitly outside the scope of a
technical security framework and require policy collaboration.

\subsection{Neurorights Framework Integration}

The \qif framework implements the four neurorights proposed by Ienca and
Andorno~\cite{ienca2017neurorights}:

\begin{enumerate}
  \item \textbf{Cognitive Liberty}: Captured by the consent violation metric (CV).
        Any technique that operates without informed consent scores CV $\geq$
        Partial.
  \item \textbf{Mental Privacy}: Captured by the cognitive integrity metric (CG).
        Techniques that decode intent, extract memories, or infer identity score
        CG $\geq$ Low.
  \item \textbf{Mental Integrity}: Captured by the biological impact metric (BI)
        and neuroplasticity metric (NP). Physical harm to neural tissue or
        induced structural changes violate mental integrity.
  \item \textbf{Psychological Continuity}: Captured by the reversibility metric
        (RV). Irreversible changes to neural function threaten the continuity of
        personal identity.
\end{enumerate}

\subsection{Regulatory Mapping}

The framework maps to existing and emerging regulatory instruments:

\begin{itemize}
  \item \textbf{FDA Section 524B}~\cite{fda524b}: Cybersecurity requirements for
        connected medical devices; NISS extends CVSS scoring as required by FDA
        premarket submissions
  \item \textbf{EU MDR 2017/745}~\cite{eumdr2017}: Medical device risk management;
        TARA provides the threat taxonomy required for conformity assessment
  \item \textbf{ISO 14971}~\cite{iso14971}: Risk management for medical devices;
        the NIC pipeline formalizes the harm pathway from technical failure to
        patient injury
  \item \textbf{HIPAA}~\cite{hipaa1996}: Health data privacy; neural data
        classification extends protected health information categories
  \item \textbf{GDPR}~\cite{gdpr2016}: Data protection; neural data as sensitive
        personal data under Article 9
  \item \textbf{Colorado Privacy Act}~\cite{colorado2024neuraldata}: First US state
        to classify neural data as sensitive personal data (2024)
\end{itemize}
