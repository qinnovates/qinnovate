% ═══════════════════════════════════════════════════════════════
% Section 1: Introduction
% ═══════════════════════════════════════════════════════════════

\section{Introduction}
\label{sec:introduction}

Brain-computer interfaces (BCIs) are no longer confined to research laboratories.
Neuralink has implanted its N1 device in human patients~\cite{musk2019neuralink},
Synchron's Stentrode has demonstrated motor neuroprosthesis via neurointerventional
surgery~\cite{oxley2021stentrode}, and Blackrock Neurotech's Utah array has enabled
high-performance speech neuroprostheses~\cite{willett2023speech}. Paradromics is
advancing toward high-bandwidth cortical interfaces for paralyzed patients.
Investment in neurotechnology grew 700\% between 2014 and 2021, with the global
market projected to reach \$25 billion by 2030~\cite{unesco2025recommendation}.

Despite this rapid commercialization, no security standard exists specifically for
neural devices. Section 3305 of the Food and Drug Omnibus Reform Act of 2022
(FDORA, Pub.~L. 117-328) added Section 524B to the FD\&C Act~\cite{fda524b},
mandating cybersecurity documentation---including threat modeling, software bills
of materials, and vulnerability disclosure---in all premarket submissions for
connected medical devices. Since October 2023, the FDA enforces a
Refuse-to-Accept policy for non-compliant submissions. However, Section 524B
does not specify which threat taxonomy or scoring system to use, and the
referenced standards (CVSS, IEC 62443, AAMI TIR57, ISO 14971) contain no
provisions for the unique risks of devices that read and write neural signals.
The EU Medical Device Regulation~\cite{eumdr2017} similarly lacks neural-specific
security requirements.

\subsection{The Scoring Gap}

The Common Vulnerability Scoring System (CVSS v4.0)~\cite{first2023cvss4} is the
industry standard for assessing vulnerability severity. CVSS excels at capturing
exploitability characteristics---attack vector, complexity, privileges required,
user interaction---and information system impact across confidentiality, integrity,
and availability. However, CVSS was designed for information technology assets.
It has no concept of:

\begin{itemize}
  \item \textbf{Biological tissue damage}---seizure induction, neural tissue
        necrosis, involuntary motor activation
  \item \textbf{Cognitive integrity}---thought privacy, perception manipulation,
        identity modification
  \item \textbf{Consent boundaries}---the distinction between operating within
        consented parameters and covert neural manipulation
  \item \textbf{Damage reversibility}---IT assets can be restored from backup;
        neural tissue cannot be rebooted
  \item \textbf{Neuroplastic consequences}---prolonged adversarial stimulation can
        cause lasting structural changes to the brain
\end{itemize}

When a vulnerability in a BCI device can induce seizures, decode private thoughts,
or cause irreversible brain damage, a CVSS base score alone is insufficient.
The gap is not theoretical: our analysis of 102 BCI attack techniques shows that
96.1\% require scoring dimensions that CVSS cannot express.

\subsection{Contributions}

This paper presents an integrated security framework with four contributions:

\begin{enumerate}
  \item \textbf{The Hourglass Architecture} (Section~\ref{sec:hourglass}): An
        11-band model mapping attack surfaces across three zones---neural (7~bands,
        from neocortex to spinal cord), interface (1~band, the electrode-tissue
        boundary), and synthetic (3~bands, from analog electronics to wireless
        radio). The hourglass shape reflects a natural security chokepoint at the
        neural interface.

  \item \textbf{TARA Threat Taxonomy} (Section~\ref{sec:tara}): A registry of
        102~techniques across 15~tactics and 8~domains, each classified by
        evidence status, severity, and dual-use therapeutic potential. TARA is
        an independent taxonomy inspired by the structural methodology of MITRE
        ATT\&CK\textsuperscript{\textregistered}, tailored to the BCI domain.

  \item \textbf{NISS Scoring System} (Section~\ref{sec:niss}): The Neural Impact
        Scoring System---a CVSS v4.0 extension designed to conform with
        FIRST.org's official extension mechanism (User Guide
        \S3.11)~\cite{first2023cvss4userguide}, adding five neural-specific
        metrics. Formal registration with FIRST.org is planned as future work.

  \item \textbf{The Neural Impact Chain} (Section~\ref{sec:nic}): A methodology
        mapping security vulnerabilities to psychiatric diagnoses through a
        six-stage pipeline: technique $\to$ hourglass band $\to$ neural
        structure $\to$ cognitive function $\to$ NISS score $\to$ DSM-5-TR
        diagnostic code~\cite{apa2022dsm5tr}.
\end{enumerate}

Sections~\ref{sec:governance}--\ref{sec:case-studies} address governance alignment
with international neuroethics frameworks and present comparative case studies.
Section~\ref{sec:limitations} discusses limitations and validation gaps honestly.
The complete framework, threat registry, and scoring system are released as open
source under the Apache 2.0 license.
