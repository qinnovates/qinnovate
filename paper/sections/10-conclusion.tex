% ═══════════════════════════════════════════════════════════════
% Section 10: Conclusion
% ═══════════════════════════════════════════════════════════════

\section{Conclusion}
\label{sec:conclusion}

Brain-computer interfaces are transitioning from laboratory prototypes to
commercial medical devices. The security frameworks designed for information
technology---while necessary---are insufficient for devices that read and write
neural signals. A vulnerability in a BCI is not merely a software bug; it is a
potential path to seizures, cognitive manipulation, privacy violation at the level
of thought, and irreversible neural harm.

This paper presented the \qif framework: an integrated system comprising an
11-band hourglass architecture, a 102-technique threat taxonomy (TARA), a
CVSS v4.0 extension for neural-specific scoring (NISS), and the Neural Impact
Chain---a first-of-its-kind methodology for mapping security vulnerabilities to
DSM-5-TR psychiatric diagnoses. Analysis of all 102 techniques reveals that
96.1\% require scoring dimensions CVSS cannot express, 75.5\% have therapeutic
dual-use analogs, and 58.8\% pose direct or indirect psychiatric diagnostic risk.

The framework has significant limitations---no empirical validation on BCI
hardware, no clinical validation of DSM-5-TR mappings, and single-author
scoring---which we have documented transparently. These are not reasons to delay
publication; they are invitations to collaborate. The BCI industry is moving
faster than security standards. Neuralink, Synchron, and Blackrock Neurotech are
implanting devices in patients today. The gap between what these devices can do
and what security frameworks can assess grows wider each month.

The complete framework, threat registry, NISS specification, and scoring data are
released as open source under the Apache 2.0 license. We invite collaboration
from the neuroscience, neuroethics, cybersecurity, and clinical psychiatry
communities. Formal registration of NISS with FIRST.org's CVSS Special Interest
Group is planned as future work.

The question is no longer whether BCI security frameworks are needed.
The question is whether they will be ready before the first patient is harmed.
