% ═══════════════════════════════════════════════════════════════
% Section 3: The Hourglass Architecture
% ═══════════════════════════════════════════════════════════════

\section{The Hourglass Architecture}
\label{sec:hourglass}

The \qif architecture maps the full BCI attack surface using an 11-band hourglass
model organized into three zones. The model is derived from two independent
design traditions: the OSI networking reference model~\cite{zimmermann1980osi},
which stratifies communication systems into functional layers, and functional
neuroanatomy, which organizes the nervous system by anatomical structure and
physiological role. The hourglass shape borrows from the Internet protocol
architecture~\cite{deering1998hourglass}, where IP serves as a narrow waist
through which all traffic must pass.

\subsection{Design Rationale}

A BCI system spans from high-level cortical computation to low-level radio
transmission. Any security framework must account for threats at every point
along this path. We observe that the neural interface---the electrode-tissue
boundary---forms a natural chokepoint analogous to IP in the Internet hourglass:
all signals, whether neural or synthetic, must cross this boundary. Above the
interface, attack surfaces expand through the complexity of neural tissue. Below
it, they expand through electronic and wireless systems. The resulting shape is
an asymmetric hourglass with the narrowest point at the interface.

\subsection{Band Definitions}

Table~\ref{tab:bands} defines all 11 bands. The model uses a 7-1-3 asymmetric
structure: seven neural bands (N7--N1), one interface band (I0), and three
synthetic bands (S1--S3).

\begin{table}[H]
\centering
\caption{QIF Hourglass: 11 bands across three zones.}
\label{tab:bands}
\small
\begin{tabularx}{\textwidth}{@{}l l l X@{}}
\toprule
\textbf{Band} & \textbf{Name} & \textbf{Zone} & \textbf{Attack Surface} \\
\midrule
N7 & Neocortex         & Neural     & PFC, M1, V1, Broca, Wernicke---executive function, language, movement, perception \\
N6 & Limbic System     & Neural     & Hippocampus, amygdala, insula---emotion, memory, interoception \\
N5 & Basal Ganglia     & Neural     & Striatum, STN, substantia nigra---motor selection, reward, habit \\
N4 & Diencephalon      & Neural     & Thalamus, hypothalamus---sensory gating, consciousness relay \\
N3 & Cerebellum        & Neural     & Cerebellar cortex, deep nuclei---motor coordination, timing \\
N2 & Brainstem         & Neural     & Medulla, pons, midbrain---vital functions, arousal, reflexes \\
N1 & Spinal Cord       & Neural     & Cervical through sacral---reflexes, peripheral relay \\
\midrule
I0 & Neural Interface  & Interface  & Electrode-tissue boundary---the hourglass waist; all signals must cross \\
\midrule
S1 & Analog / Near-Field & Synthetic & Amplification, ADC, near-field EM (0--10\,kHz) \\
S2 & Digital / Telemetry & Synthetic & Decoding, BLE/WiFi, telemetry (10\,kHz--1\,GHz) \\
S3 & Radio / Wireless    & Synthetic & RF, directed energy, application layer (1\,GHz+) \\
\bottomrule
\end{tabularx}
\end{table}

\subsection{Zone Structure}

\paragraph{Neural Zone (N7--N1).}
The neural zone encompasses all biological structures from the neocortex to the
spinal cord. Attack surfaces in this zone involve direct interaction with neural
tissue: signal injection, neural manipulation, cognitive exploitation, and
physical safety threats. The ordering follows neuroanatomical hierarchy---higher
bands represent more complex cognitive functions, while lower bands represent
more fundamental physiological processes. Attacks at lower neural bands (N1--N2)
threaten vital functions; attacks at higher bands (N6--N7) threaten cognition,
identity, and autonomy.

\paragraph{Interface Zone (I0).}
Band I0 is the singular chokepoint where biological and synthetic signals
convert. For invasive BCIs, this is the electrode-tissue boundary where
electrical signals are transduced from ionic currents in neural tissue to
electronic currents in silicon. For non-invasive devices, it is the sensor
surface (e.g., scalp electrodes for EEG). The interface band is the narrowest
point in the hourglass and the most critical for security: every neural signal
that reaches the synthetic system, and every stimulation signal that reaches
neural tissue, must pass through I0.

\paragraph{Synthetic Zone (S1--S3).}
The synthetic zone covers electronic and wireless systems from the analog
front-end through digital processing to radio transmission. S1 handles analog
signal conditioning and near-field electromagnetic effects. S2 encompasses
digital processing, Bluetooth Low Energy, WiFi, and telemetry protocols.
S3 covers radio frequency transmission, directed energy, and application-layer
communication. Threats in this zone are closest to traditional IT security
and are most amenable to conventional countermeasures.

\subsection{Security Properties}

The hourglass architecture provides three properties relevant to security analysis:

\begin{enumerate}
  \item \textbf{Completeness}: Every component of a BCI system maps to exactly
        one band. There is no attack surface that falls outside the model.

  \item \textbf{Chokepoint identification}: Band I0 identifies the natural
        monitoring point where all signals can be inspected during transit between
        neural and synthetic zones.

  \item \textbf{Threat locality}: Each technique in the TARA taxonomy
        (Section~\ref{sec:tara}) is annotated with the bands it affects, enabling
        defenders to understand the spatial extent of an attack across the
        architecture.
\end{enumerate}

Figure~\ref{fig:hourglass} illustrates the 11-band model with relative band
widths reflecting attack surface breadth at each layer.

\begin{figure}[H]
  \centering
  \includegraphics[width=0.6\textwidth]{figures/hourglass.pdf}
  \caption{The QIF Hourglass Architecture. Band widths reflect relative attack
  surface breadth. The neural interface (I0) forms the narrow waist---the
  security chokepoint through which all signals must pass.}
  \label{fig:hourglass}
\end{figure}
