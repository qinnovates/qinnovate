% ═══════════════════════════════════════════════════════════════
% Section 5: NISS — Neural Impact Scoring System
% ═══════════════════════════════════════════════════════════════

\section{NISS: Neural Impact Scoring System}
\label{sec:niss}

The Neural Impact Scoring System (NISS) is a CVSS v4.0 extension following
FIRST.org's official extension mechanism (User Guide
\S3.11)~\cite{first2023cvss4userguide}. NISS adds five metrics that capture
dimensions CVSS was never designed to express: biological tissue damage, cognitive
integrity violations, consent boundary violations, damage reversibility, and
neuroplastic consequences.

\subsection{Gap Analysis: Why CVSS Alone Is Insufficient}

We mapped all 102 TARA techniques to CVSS v4.0 base vectors and classified the
results into three gap groups based on how much information CVSS alone fails to
capture:

\begin{table}[H]
\centering
\caption{CVSS v4.0 gap analysis across 102 TARA techniques.}
\label{tab:gap}
\small
\begin{tabularx}{\textwidth}{@{}c X r l@{}}
\toprule
\textbf{Group} & \textbf{Gap Description} & \textbf{Count} & \textbf{Example} \\
\midrule
1 & CVSS captures most impact; NISS adds nuance &
    12 & Digital-only \\
2 & CVSS captures exploitability but misses half &
    28 & Mixed impact \\
3 & CVSS fundamentally cannot express primary impact &
    58 & Neural-dominant \\
\midrule
  & \textbf{Techniques needing NISS extension} & \textbf{98} & \textbf{96.1\%} \\
\bottomrule
\end{tabularx}
\end{table}

Group~3---where CVSS fundamentally cannot express the primary impact---contains
the majority of techniques (56.9\%). These are attacks where the most severe
consequence is biological tissue damage, cognitive coercion, or irreversible
neural harm---dimensions for which CVSS has no metric.

\subsection{Extension Metrics}

NISS defines five extension metrics, each with a graduated value set. The
metrics are designed to be orthogonal to CVSS base metrics: they capture impact
dimensions that exist only because the target system interfaces with biological
neural tissue.

\subsubsection{BI: Biological Impact}

Direct harm to neural tissue, organs, or physiological function. This dimension
has no equivalent in CVSS.

\begin{table}[H]
\centering
\small
\begin{tabular}{@{}l l l p{7cm}@{}}
\toprule
\textbf{Value} & \textbf{Label} & \textbf{Score} & \textbf{Description} \\
\midrule
N & None     & 0.0  & No tissue interaction or physical harm \\
L & Low      & 3.3  & Temporary discomfort, minor sensory disruption, reversible tissue stress \\
H & High     & 6.7  & Significant tissue damage, seizure induction, involuntary motor activation \\
C & Critical & 10.0 & Life-threatening or permanently disabling neural harm \\
\bottomrule
\end{tabular}
\end{table}

\subsubsection{CG: Cognitive Integrity}

Impact on thought processes, perception, memory, identity, or decision-making.
CVSS has no concept of thought privacy or cognitive autonomy.

\begin{table}[H]
\centering
\small
\begin{tabular}{@{}l l l p{7cm}@{}}
\toprule
\textbf{Value} & \textbf{Label} & \textbf{Score} & \textbf{Description} \\
\midrule
N & None     & 0.0  & No cognitive impact \\
L & Low      & 3.3  & Decoded intent partially exposed, minor perceptual distortion \\
H & High     & 6.7  & Full thought decoding, identity inference, or perception manipulation \\
C & Critical & 10.0 & Cognitive coercion, identity modification, or complete loss of cognitive autonomy \\
\bottomrule
\end{tabular}
\end{table}

\subsubsection{CV: Consent Violation}

Degree of violation of informed consent or cognitive autonomy. Ordered by
severity: covert (implicit) violations are worse than detectable (explicit) ones.

\begin{table}[H]
\centering
\small
\begin{tabular}{@{}l l l p{7cm}@{}}
\toprule
\textbf{Value} & \textbf{Label} & \textbf{Score} & \textbf{Description} \\
\midrule
N & None              & 0.0  & Operating within explicitly consented boundaries \\
P & Partial           & 3.3  & Action exceeds scope but subject retains some awareness \\
E & Explicit          & 6.7  & Direct violation of consent boundaries, but detectable \\
I & Implicit (covert) & 10.0 & Covert manipulation the patient cannot detect or refuse \\
\bottomrule
\end{tabular}
\end{table}

\subsubsection{RV: Reversibility}

Whether the damage can be undone. IT assets can be restored from backup. Neural
tissue cannot be rebooted.

\begin{table}[H]
\centering
\small
\begin{tabular}{@{}l l l p{7cm}@{}}
\toprule
\textbf{Value} & \textbf{Label} & \textbf{Score} & \textbf{Description} \\
\midrule
F & Full        & 0.0  & Effects fully reverse when attack stops \\
T & Temporary   & 3.3  & Effects reverse over hours to days \\
P & Partial     & 6.7  & Some effects permanent, some reversible \\
I & Irreversible & 10.0 & Permanent neural tissue destruction or cognitive change \\
\bottomrule
\end{tabular}
\end{table}

\subsubsection{NP: Neuroplasticity}

Whether the attack exploits or induces neuroplastic changes---the brain's ability
to rewire itself. This has no digital equivalent.

\begin{table}[H]
\centering
\small
\begin{tabular}{@{}l l l p{7cm}@{}}
\toprule
\textbf{Value} & \textbf{Label} & \textbf{Score} & \textbf{Description} \\
\midrule
N & None       & 0.0  & No neuroplastic effect \\
T & Temporary  & 5.0  & Short-term synaptic changes that decay within hours to days \\
S & Structural & 10.0 & Long-term or permanent neural pathway changes \\
\bottomrule
\end{tabular}
\end{table}

\subsection{PINS Flag}

NISS introduces the Persistent Involuntary Neural Stimulation (PINS) flag---a
binary indicator triggered when:

\begin{equation}
\text{PINS} = \begin{cases}
  \texttt{true} & \text{if } \text{BI} \geq \text{High} \;\lor\; \text{RV} = \text{Irreversible} \\
  \texttt{false} & \text{otherwise}
\end{cases}
\end{equation}

A PINS flag mandates immediate safety review regardless of overall score.
Across all 102 techniques, 31 are PINS-flagged (30.4\%).

\subsection{Scoring Formula}

The NISS score is computed as the weighted mean of the five metric scores:

\begin{equation}
\text{NISS} = \frac{w_{\text{BI}} \cdot \text{BI} + w_{\text{CG}} \cdot \text{CG} + w_{\text{CV}} \cdot \text{CV} + w_{\text{RV}} \cdot \text{RV} + w_{\text{NP}} \cdot \text{NP}}
              {w_{\text{BI}} + w_{\text{CG}} + w_{\text{CV}} + w_{\text{RV}} + w_{\text{NP}}}
\label{eq:niss}
\end{equation}

In the default profile, all weights are~1.0, yielding a simple arithmetic mean.
NISS supports four context profiles with differential weights:

\begin{itemize}
  \item \textbf{Clinical}: Emphasizes BI, RV, and NP (patient safety focus)
  \item \textbf{Research}: Emphasizes CG and CV (consent and cognition focus)
  \item \textbf{Consumer}: Balanced weights (general-purpose)
  \item \textbf{Military}: Emphasizes BI and CG (dual-use concern)
\end{itemize}

\subsection{Vector Format}

The NISS vector rides alongside the CVSS v4.0 base vector:

\begin{lstlisting}
CVSS:4.0/AV:N/AC:L/AT:N/PR:N/UI:N/
  VC:H/VI:H/VA:H/SC:N/SI:N/SA:N
NISS:1.0/BI:H/CG:C/CV:I/RV:P/NP:S
\end{lstlisting}

This dual-vector architecture means security teams can triage using familiar CVSS
scores while BCI-specific teams see the neural dimensions that determine whether a
vulnerability is a software bug or a patient safety emergency.

\subsection{NISS Severity Distribution}

Across all 102 techniques, the NISS severity distribution differs markedly from
CVSS severity:

\begin{table}[H]
\centering
\caption{NISS severity distribution (all 102 techniques).}
\label{tab:niss-severity}
\small
\begin{tabular}{@{}l r r@{}}
\toprule
\textbf{NISS Severity} & \textbf{Count} & \textbf{\%} \\
\midrule
High     & 21 & 20.6\% \\
Medium   & 29 & 28.4\% \\
Low      & 51 & 50.0\% \\
None     &  1 &  1.0\% \\
\bottomrule
\end{tabular}
\end{table}

The NISS distribution is more uniform than CVSS because NISS captures impact
dimensions that CVSS flattens into high/critical. Techniques rated ``high'' by
CVSS may distribute across low, medium, and high NISS scores depending on whether
they involve biological tissue damage or are purely digital.

Figure~\ref{fig:niss-gap} visualizes the gap between CVSS and NISS scoring.

\begin{figure}[H]
  \centering
  \includegraphics[width=0.7\textwidth]{figures/niss-gap.pdf}
  \caption{CVSS v4.0 vs.\ NISS gap analysis: 96.1\% of techniques require
  extension metrics CVSS cannot express.}
  \label{fig:niss-gap}
\end{figure}
